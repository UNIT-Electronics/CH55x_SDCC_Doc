%% Generated by Sphinx.
\def\sphinxdocclass{report}
\documentclass[letterpaper,10pt,english]{sphinxmanual}
\ifdefined\pdfpxdimen
   \let\sphinxpxdimen\pdfpxdimen\else\newdimen\sphinxpxdimen
\fi \sphinxpxdimen=.75bp\relax
\ifdefined\pdfimageresolution
    \pdfimageresolution= \numexpr \dimexpr1in\relax/\sphinxpxdimen\relax
\fi
%% let collapsible pdf bookmarks panel have high depth per default
\PassOptionsToPackage{bookmarksdepth=5}{hyperref}

\PassOptionsToPackage{booktabs}{sphinx}
\PassOptionsToPackage{colorrows}{sphinx}

\PassOptionsToPackage{warn}{textcomp}
\usepackage[utf8]{inputenc}
\ifdefined\DeclareUnicodeCharacter
% support both utf8 and utf8x syntaxes
  \ifdefined\DeclareUnicodeCharacterAsOptional
    \def\sphinxDUC#1{\DeclareUnicodeCharacter{"#1}}
  \else
    \let\sphinxDUC\DeclareUnicodeCharacter
  \fi
  \sphinxDUC{00A0}{\nobreakspace}
  \sphinxDUC{2500}{\sphinxunichar{2500}}
  \sphinxDUC{2502}{\sphinxunichar{2502}}
  \sphinxDUC{2514}{\sphinxunichar{2514}}
  \sphinxDUC{251C}{\sphinxunichar{251C}}
  \sphinxDUC{2572}{\textbackslash}
\fi
\usepackage{cmap}
\usepackage[T1]{fontenc}
\usepackage{amsmath,amssymb,amstext}
\usepackage{babel}



\usepackage{tgtermes}
\usepackage{tgheros}
\renewcommand{\ttdefault}{txtt}



\usepackage[Bjarne]{fncychap}
\usepackage[,numfigreset=1,mathnumfig]{sphinx}

\fvset{fontsize=auto}
\usepackage{geometry}


% Include hyperref last.
\usepackage{hyperref}
% Fix anchor placement for figures with captions.
\usepackage{hypcap}% it must be loaded after hyperref.
% Set up styles of URL: it should be placed after hyperref.
\urlstyle{same}

\addto\captionsenglish{\renewcommand{\contentsname}{Contents}}

\usepackage{sphinxmessages}
\setcounter{tocdepth}{1}



\title{Cocket Nova Development Board Programming Guide C/C++}
\date{Jul 10, 2024}
\release{0.1.0}
\author{Cesar Bautista}
\newcommand{\sphinxlogo}{\sphinxincludegraphics{Logo-UNIT_Web-04-800x182.png}\par}
\renewcommand{\releasename}{Release}
\makeindex
\begin{document}

\ifdefined\shorthandoff
  \ifnum\catcode`\=\string=\active\shorthandoff{=}\fi
  \ifnum\catcode`\"=\active\shorthandoff{"}\fi
\fi

\pagestyle{empty}
\sphinxmaketitle
\pagestyle{plain}
\sphinxtableofcontents
\pagestyle{normal}
\phantomsection\label{\detokenize{index::doc}}


\begin{sphinxadmonition}{note}{Note:}
\sphinxAtStartPar
This project is under active development.
\end{sphinxadmonition}

\sphinxAtStartPar
The Coket Nova guide is an instructional manual for utilizing the compiler SDCC. It serves various purposes, allowing users to explore different features and obtain new projects and
configurations.
The Cocket Nova CH552 boards are incredibly easy to use. The projects focus on innovation and
present various alternatives for usage.

\begin{figure}[htbp]
\centering
\capstart

\noindent\sphinxincludegraphics[width=0.800\linewidth]{{CH552_Sq}.png}
\caption{Cocket Nova CH552 Desing graphic}\label{\detokenize{index:id1}}\label{\detokenize{index:ch552}}\end{figure}

\begin{sphinxadmonition}{tip}{Tip:}
\sphinxAtStartPar
The codes examples in github are available in the following link: \sphinxhref{https://github.com/UNIT-Electronics/CH55x\_SDCC\_Examples}{CH55x\_SDCC\_Examples Github}
\end{sphinxadmonition}

\sphinxstepscope


\chapter{Cocket Nova Developmet Board CH552 Guide}
\label{\detokenize{about:cocket-nova-developmet-board-ch552-guide}}\label{\detokenize{about::doc}}
\sphinxAtStartPar
This is an excellent guide for beginner programmers, focused on using the SDCC compiler in both Windows and Linux environments.
Here, you can find excellent references and examples along with comprehensive documentation focusing on developing technology for embedded systems.
This course covers everything from installation and setting up the compiler to managing project dependencies and developing code.
It’s a valuable resource that will guide you through the development process using high\sphinxhyphen{}quality technology, ensuring long\sphinxhyphen{}lasting and robust projects.


\section{Why Use the SDCC Compiler?}
\label{\detokenize{about:why-use-the-sdcc-compiler}}
\sphinxAtStartPar
The Small Device C Compiler (SDCC) is a highly regarded tool in the field of embedded systems development. Here are several reasons why you might choose to use the SDCC compiler for your projects:
\begin{enumerate}
\sphinxsetlistlabels{\arabic}{enumi}{enumii}{}{.}%
\item {} 
\sphinxAtStartPar
\sphinxstylestrong{Free and Open\sphinxhyphen{}Source}: SDCC is freely available and open\sphinxhyphen{}source, which means you can use it without licensing fees and contribute to its development if you wish.

\item {} 
\sphinxAtStartPar
\sphinxstylestrong{Wide Microcontroller Support}: It supports a broad range of microcontrollers, including popular ones like the CH552, making it a versatile choice for various projects.

\item {} 
\sphinxAtStartPar
\sphinxstylestrong{Ease of Use}: SDCC is known for its user\sphinxhyphen{}friendly interface and straightforward setup, which helps developers get started quickly.

\item {} 
\sphinxAtStartPar
\sphinxstylestrong{Active Community and Documentation}: With an active community and extensive documentation, you can find support and resources to help you solve any issues you encounter.

\item {} 
\sphinxAtStartPar
\sphinxstylestrong{Compatibility}: SDCC is compatible with many other tools and environments, allowing for seamless integration into existing workflows.

\end{enumerate}


\section{Understanding Programming Languages in Embedded Systems}
\label{\detokenize{about:understanding-programming-languages-in-embedded-systems}}
\sphinxAtStartPar
To develop effective embedded systems, it’s crucial to understand the different types of programming languages used:


\subsection{Machine Language}
\label{\detokenize{about:machine-language}}
\sphinxAtStartPar
Machine language, also known as machine code, is the most fundamental level of programming. Instructions are written in binary bit patterns, which are combinations of 1s and 0s. These patterns correspond to HIGH and LOW voltage levels that the microcontroller or microprocessor can directly interpret. This language is the most difficult for humans to use due to its complexity and lack of readability.


\subsection{Assembly Language}
\label{\detokenize{about:assembly-language}}
\sphinxAtStartPar
Assembly language is a step above machine language, providing a more human\sphinxhyphen{}readable format. It uses mnemonics and hexadecimal codes to represent machine language instructions. For instance, the 8051 microcontroller assembly language includes a combination of English\sphinxhyphen{}like words called mnemonics and hexadecimal numbers. Despite being more readable than machine language, it still requires an in\sphinxhyphen{}depth understanding of the microcontroller’s architecture.


\subsection{High\sphinxhyphen{}level Language}
\label{\detokenize{about:high-level-language}}
\sphinxAtStartPar
High\sphinxhyphen{}level languages simplify programming by abstracting away the intricate details of the microcontroller’s architecture. These languages use familiar words and statements, making them easier to learn and use. Examples of high\sphinxhyphen{}level languages include BASIC, C, Pascal, C++, and Java. Programs written in high\sphinxhyphen{}level languages are translated into machine code by a compiler, bridging the gap between human\sphinxhyphen{}friendly code and machine\sphinxhyphen{}understandable instructions.

\sphinxAtStartPar
By understanding these \sphinxhref{https://gmostofabd.github.io/8051-Assembly-Programming/}{different levels of programming languages} , you can choose the most appropriate one for your project, balancing ease of use with the level of control you need over the hardware.


\section{Requirements}
\label{\detokenize{about:requirements}}\begin{itemize}
\item {} 
\sphinxAtStartPar
\sphinxhref{https://www.python.org/downloads/}{Python}  (Package Installation and Environments)

\item {} 
\sphinxAtStartPar
Utilization of Operating System Controllers

\item {} 
\sphinxAtStartPar
Understanding of Basic Electronics

\end{itemize}

\begin{sphinxadmonition}{tip}{Tip:}
\sphinxAtStartPar
This  guide is tailored for individuals with a basic understanding of programming and electronics. It’s ideal for those interested in diving into embedded systems and programming microcontrollers.
\end{sphinxadmonition}


\section{PinOut}
\label{\detokenize{about:pinout}}
\sphinxAtStartPar
The CH552 microcontroller development board has a total of 16 pins, each with a specific function. The following is a list of the pins and their respective functions:

\begin{figure}[htbp]
\centering
\capstart

\noindent\sphinxincludegraphics[width=0.800\linewidth]{{PinOut_CH552}.jpg}
\caption{Cocket Nova CH552 PinOut}\label{\detokenize{about:id2}}\label{\detokenize{about:id1}}\end{figure}

\sphinxstepscope


\chapter{Environment Setup on Ubuntu}
\label{\detokenize{install_linux:environment-setup-on-ubuntu}}\label{\detokenize{install_linux::doc}}
\begin{sphinxadmonition}{caution}{Caution:}
\sphinxAtStartPar
This project is under active development. The information provided here is subject to change.
\end{sphinxadmonition}

\sphinxAtStartPar
Update the operating system:

\begin{sphinxVerbatim}[commandchars=\\\{\}]
\PYG{n}{sudo} \PYG{n}{apt} \PYG{n}{update}
\end{sphinxVerbatim}

\sphinxAtStartPar
Install \sphinxtitleref{make}, \sphinxtitleref{binutils} and \sphinxtitleref{sdcc}:

\begin{sphinxVerbatim}[commandchars=\\\{\}]
\PYG{n}{sudo} \PYG{n}{apt} \PYG{n}{install} \PYG{n}{make}
\PYG{n}{sudo} \PYG{n}{apt} \PYG{n}{install} \PYG{n}{binutils}
\PYG{n}{sudo} \PYG{n}{apt} \PYG{n}{install} \PYG{n}{sdcc}
\end{sphinxVerbatim}

\sphinxAtStartPar
Clone the examples with the main code:

\begin{sphinxVerbatim}[commandchars=\\\{\}]
\PYG{n}{git} \PYG{n}{clone} \PYG{n}{https}\PYG{p}{:}\PYG{o}{/}\PYG{o}{/}\PYG{n}{github}\PYG{o}{.}\PYG{n}{com}\PYG{o}{/}\PYG{n}{UNIT}\PYG{o}{\PYGZhy{}}\PYG{n}{Electronics}\PYG{o}{/}\PYG{n}{CH55x\PYGZus{}SDCC\PYGZus{}Examples}\PYG{o}{.}\PYG{n}{git}
\end{sphinxVerbatim}

\sphinxAtStartPar
Navigate to the path:

\begin{sphinxVerbatim}[commandchars=\\\{\}]
\PYG{n}{cd} \PYG{n}{CH55x\PYGZus{}SDCC\PYGZus{}Examples}\PYG{o}{/}\PYG{n}{Software}\PYG{o}{/}\PYG{n}{examples}\PYG{o}{/}\PYG{n}{Blink}
\end{sphinxVerbatim}

\sphinxAtStartPar
Execute the command:

\begin{sphinxVerbatim}[commandchars=\\\{\}]
\PYG{n}{make} \PYG{n}{help}
\end{sphinxVerbatim}

\sphinxAtStartPar
You will see:

\begin{sphinxVerbatim}[commandchars=\\\{\}]
\PYG{n}{Use} \PYG{n}{the} \PYG{n}{following} \PYG{n}{commands}\PYG{p}{:}
\PYG{n}{make} \PYG{n+nb}{all}     \PYG{n+nb}{compile}\PYG{p}{,} \PYG{n}{build}\PYG{p}{,} \PYG{o+ow}{and} \PYG{n}{keep} \PYG{n+nb}{all} \PYG{n}{files}
\PYG{n}{make} \PYG{n+nb}{hex}     \PYG{n+nb}{compile} \PYG{o+ow}{and} \PYG{n}{build} \PYG{n}{blink}\PYG{o}{.}\PYG{n}{hex}
\PYG{n}{make} \PYG{n+nb}{bin}     \PYG{n+nb}{compile} \PYG{o+ow}{and} \PYG{n}{build} \PYG{n}{blink}\PYG{o}{.}\PYG{n}{bin}
\PYG{n}{make} \PYG{n}{clean}   \PYG{n}{remove} \PYG{n+nb}{all} \PYG{n}{build} \PYG{n}{files}
\end{sphinxVerbatim}

\sphinxAtStartPar
Connect a device with the BOOT button pressed:

\begin{sphinxVerbatim}[commandchars=\\\{\}]
\PYG{n}{lsusb}
\end{sphinxVerbatim}

\sphinxAtStartPar
The device will be shown with this description:

\begin{sphinxVerbatim}[commandchars=\\\{\}]
pc@LAPTOP:\PYGZti{}\PYGZdl{} lsusb
Bus 002 Device 001: ID 1d6b:0003 Linux Foundation 3.0 root hub
Bus 001 Device 002: ID 4348:55e0 WinChipHead
Bus 001 Device 001: ID 1d6b:0002 Linux Foundation 2.0 root hub
\end{sphinxVerbatim}

\sphinxstepscope


\chapter{Environment Setup on Windows}
\label{\detokenize{install_windows:environment-setup-on-windows}}\label{\detokenize{install_windows::doc}}
\sphinxAtStartPar
This section provides a step\sphinxhyphen{}by\sphinxhyphen{}step guide to setting up the SDCC compiler on Windows operating systems. It also includes instructions for installing the necessary tools and configuring the environment variables. Additionally, it covers the installation of the pyusb library and updating the driver using Zadig.


\section{Compiler Installation}
\label{\detokenize{install_windows:compiler-installation}}
\sphinxAtStartPar
Follow these steps to install the necessary tools:
\begin{enumerate}
\sphinxsetlistlabels{\arabic}{enumi}{enumii}{}{.}%
\item {} 
\sphinxAtStartPar
\sphinxstylestrong{Install Git for Windows}
Download and install Git for Windows from the \sphinxhref{https://git-scm.com/download}{official Git website}.

\item {} 
\sphinxAtStartPar
\sphinxstylestrong{Install SDCC}
Download and install the latest version of SDCC from the \sphinxhref{https://sourceforge.net/projects/sdcc/}{SDCC downloads page}.

\item {} 
\sphinxAtStartPar
\sphinxstylestrong{Install MinGW}
Download and install MinGW, a set of development tools for Windows, from the \sphinxhref{https://sourceforge.net/projects/mingw/}{official MinGW website}.

\item {} 
\sphinxAtStartPar
\sphinxstylestrong{Install CH372 Driver}
Download the latest version of the CH372 driver from the \sphinxhref{https://www.wch-ic.com/downloads/CH372DRV\_EXE.html}{official website}.

\item {} 
\sphinxAtStartPar
\sphinxstylestrong{Install Zadig}
Download and install the latest version of Zadig from the \sphinxhref{https://zadig.akeo.ie/}{official website}.

\item {} 
\sphinxAtStartPar
\sphinxstylestrong{Install Filter Wizard}
Download the latest version of libusb\sphinxhyphen{}win32 from the \sphinxhref{https://sourceforge.net/projects/libusb-win32/files/libusb-win32-releases/1.2.7.3/}{official website}. This driver is used by the Loadupch tool to communicate with the device.

\item {} 
\sphinxAtStartPar
\sphinxstylestrong{Install Python}
Download and install the latest version of Python from the \sphinxhref{https://www.python.org/downloads/}{official Python website}.

\end{enumerate}

\begin{sphinxadmonition}{tip}{Tip:}
\sphinxAtStartPar
It is recommended to install the tools in the order listed above.
\end{sphinxadmonition}

\begin{sphinxadmonition}{caution}{Caution:}
\sphinxAtStartPar
Remember to restart your computer after installing the tools.
\end{sphinxadmonition}


\section{Environment Variable Configuration}
\label{\detokenize{install_windows:environment-variable-configuration}}
\sphinxAtStartPar
Remember that for Windows operating systems, an extra step is necessary, which is to open the environment variable \sphinxhyphen{}\textgreater{} Edit environment variable:

\begin{sphinxVerbatim}[commandchars=\\\{\}]
\PYG{n}{C}\PYG{p}{:}\PYGZbs{}\PYG{n}{MinGW}\PYGZbs{}\PYG{n+nb}{bin}
\end{sphinxVerbatim}


\section{Locate the file}
\label{\detokenize{install_windows:locate-the-file}}
\sphinxAtStartPar
After installing MinGW, you will need to locate the \sphinxtitleref{mingw32\sphinxhyphen{}make.exe} file. This file is typically found in the \sphinxtitleref{C:MinGWbin} directory. Once located, rename the file to \sphinxtitleref{make.exe}.

\begin{figure}[htbp]
\centering
\capstart

\noindent\sphinxincludegraphics[width=0.900\linewidth]{{make_file}.png}
\caption{Locating the \sphinxtitleref{mingw32\sphinxhyphen{}make.exe} file}\label{\detokenize{install_windows:id3}}\label{\detokenize{install_windows:make-file}}\end{figure}


\section{Rename it}
\label{\detokenize{install_windows:rename-it}}
\sphinxAtStartPar
After locating \sphinxtitleref{mingw32\sphinxhyphen{}make.exe}, rename it to \sphinxtitleref{make.exe}. This change is necessary for compatibility with many build scripts that expect the command to be named \sphinxtitleref{make}.

\begin{figure}[htbp]
\centering
\capstart

\noindent\sphinxincludegraphics[width=0.900\linewidth]{{rename}.png}
\caption{Renaming \sphinxtitleref{mingw32\sphinxhyphen{}make.exe} to \sphinxtitleref{make.exe}}\label{\detokenize{install_windows:id4}}\label{\detokenize{install_windows:rename}}\end{figure}

\begin{sphinxadmonition}{warning}{Warning:}
\sphinxAtStartPar
If you encounter any issues, create a copy of the file and then rename the copy to \sphinxtitleref{make.exe}.
\end{sphinxadmonition}


\section{Add the path to the environment variable}
\label{\detokenize{install_windows:add-the-path-to-the-environment-variable}}
\sphinxAtStartPar
Next, you need to add the path to the MinGW bin directory to your system’s environment variables. This allows the \sphinxtitleref{make} command to be recognized from any command prompt.
\begin{enumerate}
\sphinxsetlistlabels{\arabic}{enumi}{enumii}{}{.}%
\item {} 
\sphinxAtStartPar
Open the Start Search, type in “env”, and select “Edit the system environment variables”.

\item {} 
\sphinxAtStartPar
In the System Properties window, click on the “Environment Variables” button.

\item {} 
\sphinxAtStartPar
In the Environment Variables window, under “System variables”, select the “Path” variable and click “Edit”.

\item {} 
\sphinxAtStartPar
In the Edit Environment Variable window, click “New” and add the path:

\begin{sphinxVerbatim}[commandchars=\\\{\}]
\PYG{n}{C}\PYG{p}{:}\PYGZbs{}\PYG{n}{MinGW}\PYGZbs{}\PYG{n+nb}{bin}
\end{sphinxVerbatim}

\end{enumerate}

\begin{figure}[htbp]
\centering
\capstart

\noindent\sphinxincludegraphics[width=0.600\linewidth]{{var_env}.png}
\caption{Adding MinGW bin directory to environment variables}\label{\detokenize{install_windows:id5}}\label{\detokenize{install_windows:var-env}}\end{figure}


\section{Verify the installation}
\label{\detokenize{install_windows:verify-the-installation}}
\sphinxAtStartPar
To verify that the \sphinxtitleref{make} command is correctly set up, open a new command prompt and type:

\begin{sphinxVerbatim}[commandchars=\\\{\}]
\PYG{n}{make} \PYG{o}{\PYGZhy{}}\PYG{o}{\PYGZhy{}}\PYG{n}{version}
\end{sphinxVerbatim}

\sphinxAtStartPar
You should see the version information for \sphinxtitleref{make}, indicating that it is correctly installed and recognized by the system.

\begin{figure}[htbp]
\centering
\capstart

\noindent\sphinxincludegraphics[width=0.900\linewidth]{{make_version}.png}
\caption{Verifying the installation of \sphinxtitleref{make}}\label{\detokenize{install_windows:id6}}\label{\detokenize{install_windows:verify}}\end{figure}


\section{Update driver}
\label{\detokenize{install_windows:update-driver}}
\sphinxAtStartPar
The current loading tool can utilize the default driver and coexist with the official WCHISPTool. In case the driver encounters issues, it is advisable to switch the driver version to libusb\sphinxhyphen{}win32 using \sphinxhref{https://zadig.akeo.ie/}{Zadig}.

\begin{figure}[htbp]
\centering
\capstart

\noindent\sphinxincludegraphics[width=1.000\linewidth]{{driver}.png}
\caption{driver}\label{\detokenize{install_windows:id7}}\label{\detokenize{install_windows:driver}}\end{figure}

\begin{sphinxadmonition}{warning}{Warning:}
\sphinxAtStartPar
The use of Zadig is at your own risk. if you are not familiar with the tool, it is recommended to seek assistance from someone who is. In the case of changing the driver any device  , it is important to have the original driver available to revert the changes.
\end{sphinxadmonition}

\sphinxstepscope


\chapter{Compile and Flash CH55x with SDCC}
\label{\detokenize{compile:compile-and-flash-ch55x-with-sdcc}}\label{\detokenize{compile::doc}}

\section{Running a Program}
\label{\detokenize{compile:running-a-program}}
\sphinxAtStartPar
To run the program, you will need to use a bash terminal. Follow these steps to clone the examples repository, navigate to the appropriate directory, and compile the project.
\begin{enumerate}
\sphinxsetlistlabels{\arabic}{enumi}{enumii}{}{.}%
\item {} 
\sphinxAtStartPar
\sphinxstylestrong{Clone the Examples Repository}

\sphinxAtStartPar
Begin by cloning the examples repository which contains the main code. Open your bash terminal and execute the following command:

\begin{sphinxVerbatim}[commandchars=\\\{\}]
git\PYG{+w}{ }clone\PYG{+w}{ }https://github.com/UNIT\PYGZhy{}Electronics/CH55x\PYGZus{}SDCC\PYGZus{}Examples
\end{sphinxVerbatim}

\item {} 
\sphinxAtStartPar
\sphinxstylestrong{Navigate to the Example Path}

\sphinxAtStartPar
Once the repository is cloned, navigate to the path where the example programs are located. Use the following command to change the directory:

\begin{sphinxVerbatim}[commandchars=\\\{\}]
\PYG{n+nb}{cd}\PYG{+w}{ }\PYGZti{}/CH55x\PYGZus{}SDCC\PYGZus{}Examples/Software/examples/Blink/
\end{sphinxVerbatim}

\item {} 
\sphinxAtStartPar
\sphinxstylestrong{Connect the Device}

\sphinxAtStartPar
Connect your CH55x device to your computer. Ensure that you press and hold the BOOT button while connecting the device. This is essential for the device to enter programming mode.

\item {} 
\sphinxAtStartPar
\sphinxstylestrong{Compile the Project}

\sphinxAtStartPar
To compile the project and generate the necessary files, execute the following command in your terminal:

\begin{sphinxVerbatim}[commandchars=\\\{\}]
make\PYG{+w}{ }all
\end{sphinxVerbatim}

\sphinxAtStartPar
This command will compile the project, resulting in the generation of files with various extensions necessary for flashing the microcontroller.

\end{enumerate}

\begin{figure}[htbp]
\centering
\capstart

\noindent\sphinxincludegraphics[width=0.800\linewidth]{{files}.png}
\caption{Compilation output files}\label{\detokenize{compile:id18}}\label{\detokenize{compile:files}}\end{figure}


\section{Install pyusb}
\label{\detokenize{compile:install-pyusb}}
\sphinxAtStartPar
pyusb is a Python module necessary for flashing the CH55x microcontroller. To install pyusb, follow these steps:

\sphinxAtStartPar
Install \sphinxtitleref{pyusb} on using pip

\begin{sphinxVerbatim}[commandchars=\\\{\}]
python3\PYG{+w}{ }\PYGZhy{}m\PYG{+w}{ }pip\PYG{+w}{ }install\PYG{+w}{ }pyusb
\end{sphinxVerbatim}

\sphinxAtStartPar
Then verify the installation

\begin{sphinxVerbatim}[commandchars=\\\{\}]
python3\PYG{+w}{ }\PYGZhy{}m\PYG{+w}{ }pip\PYG{+w}{ }show\PYG{+w}{ }pyusb
\end{sphinxVerbatim}

\sphinxAtStartPar
For Windows, you can use the following command:

\begin{sphinxVerbatim}[commandchars=\\\{\}]
pip\PYG{+w}{ }install\PYG{+w}{ }pyusb
\end{sphinxVerbatim}


\section{Error with pip}
\label{\detokenize{compile:error-with-pip}}
\sphinxAtStartPar
If you encounter this error, we recommend installing the Python environment:

\begin{sphinxVerbatim}[commandchars=\\\{\}]
\PYG{n}{sudo} \PYG{n}{apt} \PYG{n}{install} \PYG{n}{python3}\PYG{o}{\PYGZhy{}}\PYG{n}{venv}
\end{sphinxVerbatim}

\sphinxAtStartPar
Create an environment:

\begin{sphinxVerbatim}[commandchars=\\\{\}]
\PYG{n}{python3} \PYG{o}{\PYGZhy{}}\PYG{n}{m} \PYG{n}{venv} \PYG{o}{.}\PYG{n}{venv}
\end{sphinxVerbatim}

\sphinxAtStartPar
Activate the environment:

\begin{sphinxVerbatim}[commandchars=\\\{\}]
\PYG{n}{source} \PYG{o}{.}\PYG{n}{venv}\PYG{o}{/}\PYG{n+nb}{bin}\PYG{o}{/}\PYG{n}{activate}
\end{sphinxVerbatim}

\sphinxAtStartPar
And install \sphinxtitleref{pyusb}:

\begin{sphinxVerbatim}[commandchars=\\\{\}]
\PYG{n}{pip} \PYG{n}{install} \PYG{n}{pyusb}
\end{sphinxVerbatim}


\section{Flashing the Program}
\label{\detokenize{compile:flashing-the-program}}
\begin{sphinxuseclass}{sphinx-tabs}
\sphinxAtStartPar
\sphinxstylestrong{Arduino IDE}

\sphinxAtStartPar
Arduino IDE is a popular development environment for programming microcontrollers.
You can use the Arduino IDE to program the CH55x microcontrollers by following these steps:
\begin{enumerate}
\sphinxsetlistlabels{\arabic}{enumi}{enumii}{}{.}%
\item {} 
\sphinxAtStartPar
\sphinxstylestrong{Install Arduino IDE}

\sphinxAtStartPar
Download and install the \sphinxhref{https://www.arduino.cc/en/software}{Arduino IDE} on your computer.

\item {} 
\sphinxAtStartPar
\sphinxstylestrong{Install CH55x Board Support}

\sphinxAtStartPar
Open the Arduino IDE and navigate to \sphinxcode{\sphinxupquote{File \textgreater{} Preferences}}. In the Additional Boards Manager URLs field, add the following URL:

\begin{sphinxVerbatim}[commandchars=\\\{\}]
https://raw.githubusercontent.com/Cesarbautista10/Uelectronics\PYGZhy{}CH552\PYGZhy{}Arduino\PYGZhy{}Package\PYGZhy{}v3/main/package\PYGZus{}duino\PYGZus{}mcs51\PYGZus{}index.json
\end{sphinxVerbatim}

\item {} 
\sphinxAtStartPar
\sphinxstylestrong{Install CH55x Board}

\sphinxAtStartPar
Go to \sphinxcode{\sphinxupquote{Tools \textgreater{} Board \textgreater{} Boards Manager}} and search for \sphinxcode{\sphinxupquote{Cocket Nova}}. Install the CH55x board support package.

\end{enumerate}

\begin{sphinxadmonition}{note}{Note:}
\sphinxAtStartPar
Requires the \sphinxhref{https://www.wch-ic.com/downloads/CH372DRV\_EXE.html}{ch372} driver to be installed.
\end{sphinxadmonition}

\sphinxAtStartPar
\sphinxstylestrong{WCHISPTool}

\sphinxAtStartPar
The WCHISPTool is an official programming tool provided by WCH. It is a Windows\sphinxhyphen{}based tool that allows users to flash firmware onto CH55x microcontrollers.
To use the WCHISPTool, follow these steps:
\begin{enumerate}
\sphinxsetlistlabels{\arabic}{enumi}{enumii}{}{.}%
\item {} 
\sphinxAtStartPar
\sphinxstylestrong{Download the WCHISPTool}

\sphinxAtStartPar
Download the \sphinxhref{https://www.wch-ic.com/downloads/WCHISPTool\_Setup\_exe.html}{WCHISPTool} from the official WCH website.

\item {} 
\sphinxAtStartPar
\sphinxstylestrong{Install the WCHISPTool}

\sphinxAtStartPar
Install the WCHISPTool on your Windows computer by following the on\sphinxhyphen{}screen instructions.

\item {} 
\sphinxAtStartPar
\sphinxstylestrong{Connect the Device}

\sphinxAtStartPar
Connect your CH55x device to your computer using a USB cable. Ensure that the BOOT button is pressed while connecting the device.

\item {} 
\sphinxAtStartPar
\sphinxstylestrong{Flash the Program}

\sphinxAtStartPar
Open the WCHISPTool and select the appropriate firmware file. Click the “Download” button to flash the program onto the microcontroller.

\end{enumerate}

\begin{sphinxadmonition}{note}{Note:}
\sphinxAtStartPar
The WCHISPTool is a Windows\sphinxhyphen{}based tool and may not be compatible with other operating systems.
\end{sphinxadmonition}

\begin{figure}[htbp]
\centering
\capstart

\noindent\sphinxincludegraphics[width=0.800\linewidth]{{wchisptool}.png}
\caption{WCHISPTool interface}\label{\detokenize{compile:id19}}\label{\detokenize{compile:id6}}\end{figure}

\begin{sphinxadmonition}{warning}{Warning:}
\sphinxAtStartPar
The WCHISPTool is an official tool provided by WCH and may have limitations compared to other flashing methods.
\end{sphinxadmonition}

\begin{sphinxadmonition}{note}{Note:}
\sphinxAtStartPar
Requires the \sphinxhref{https://www.wch-ic.com/downloads/CH372DRV\_EXE.html}{ch372} driver to be installed.
\end{sphinxadmonition}

\sphinxAtStartPar
\sphinxstylestrong{chprog.py}

\sphinxAtStartPar
\sphinxstylestrong{Project:} chprog \sphinxhyphen{} Programming Tool for CH55x Microcontrollers

\sphinxAtStartPar
\sphinxstylestrong{Version:} v1.2 (2022)

\sphinxAtStartPar
\sphinxstylestrong{Author:} Stefan Wagner

\sphinxAtStartPar
\sphinxstylestrong{GitHub:} \sphinxhref{https://github.com/wagiminator}{wagiminator}

\sphinxAtStartPar
\sphinxstylestrong{License:} MIT License

\sphinxAtStartPar
\sphinxstylestrong{Description:}
Developed chprog, a Python tool for easily flashing CH55x series microcontrollers with bootloader versions 1.x and 2.x.x.

\begin{sphinxadmonition}{caution}{Caution:}
\sphinxAtStartPar
Support available up to bootloader version 2.4.0, windows only.
\end{sphinxadmonition}

\sphinxAtStartPar
\sphinxstylestrong{References:}
Inspired by and based on chflasher and wchprog by Aaron Christophel and Julius Wang:
\begin{itemize}
\item {} 
\sphinxAtStartPar
\sphinxhref{https://ATCnetz.de}{ATCnetz}

\item {} 
\sphinxAtStartPar
\sphinxhref{https://github.com/atc1441/chflasher}{chflasher on GitHub}

\item {} 
\sphinxAtStartPar
\sphinxhref{https://github.com/juliuswwj/wchprog}{wchprog on GitHub}

\end{itemize}

\sphinxAtStartPar
Once the project is compiled, you need to flash the program onto the CH55x device. Follow these steps:
\begin{enumerate}
\sphinxsetlistlabels{\arabic}{enumi}{enumii}{}{.}%
\item {} 
\sphinxAtStartPar
\sphinxstylestrong{Connect the Device}

\sphinxAtStartPar
Ensure your CH55x device is connected and the BOOT button is pressed, as done during the compilation step.

\item {} 
\sphinxAtStartPar
\sphinxstylestrong{Flash the Program}

\sphinxAtStartPar
Execute the following command to flash the compiled program onto the microcontroller:

\begin{sphinxVerbatim}[commandchars=\\\{\}]
python\PYG{+w}{ }../../tools/chprog.py\PYG{+w}{  }main.bin
\end{sphinxVerbatim}

\end{enumerate}

\begin{figure}[htbp]
\centering
\capstart

\noindent\sphinxincludegraphics[width=0.800\linewidth]{{led}.png}
\caption{LED blinking effect}\label{\detokenize{compile:id20}}\label{\detokenize{compile:id12}}\end{figure}

\begin{sphinxadmonition}{note}{Note:}
\sphinxAtStartPar
Requires the \sphinxtitleref{libusb\sphinxhyphen{}win32} driver to be installed using Zadig.
\end{sphinxadmonition}

\sphinxAtStartPar
\sphinxstylestrong{Loadupch}

\sphinxAtStartPar
The \sphinxhref{https://pypi.org/project/loadupch/}{Loadupch} is a software development prototype designed to facilitate the uploading of code to the CH552 microcontroller.
It is a user\sphinxhyphen{}friendly tool that provides a graphical interface, making it easier for users to upload their code.
Based on chprog, Loadupch is a Python tool that simplifies the process of flashing CH55x series microcontrollers with bootloader versions 1.x and 2.x.x.

\begin{sphinxadmonition}{caution}{Caution:}
\sphinxAtStartPar
Support available up to bootloader version 2.4.0, windows only.
\end{sphinxadmonition}

\begin{figure}[htbp]
\centering
\capstart

\noindent\sphinxincludegraphics[width=0.500\linewidth]{{loadupch}.png}
\caption{Loadupch tool interface}\label{\detokenize{compile:id21}}\label{\detokenize{compile:id16}}\end{figure}

\begin{sphinxadmonition}{warning}{Warning:}
\sphinxAtStartPar
The Loadupch tool is currently under development and may contain bugs. Use it at your own risk.
\end{sphinxadmonition}

\sphinxAtStartPar
To install the Loadupch tool, you can use \sphinxtitleref{pypi}. Follow these steps:
\begin{enumerate}
\sphinxsetlistlabels{\arabic}{enumi}{enumii}{}{.}%
\item {} 
\sphinxAtStartPar
\sphinxstylestrong{Install Loadupch}

\sphinxAtStartPar
Use the following command to install the \sphinxhref{https://github.com/UNIT-Electronics/ue\_loadupch\_Loader\_Firmware-}{Loadupch}. tool via pip:

\begin{sphinxVerbatim}[commandchars=\\\{\}]
pip\PYG{+w}{ }install\PYG{+w}{ }loadupch
\end{sphinxVerbatim}

\item {} 
\sphinxAtStartPar
\sphinxstylestrong{Run Loadupch}

\sphinxAtStartPar
After installation, you can run the Loadupch tool with the following command:

\begin{sphinxVerbatim}[commandchars=\\\{\}]
python\PYG{+w}{ }\PYGZhy{}m\PYG{+w}{ }loadupch
\end{sphinxVerbatim}

\begin{sphinxadmonition}{caution}{Caution:}
\sphinxAtStartPar
To recognize the device, you only need to install the \sphinxcode{\sphinxupquote{libusb\sphinxhyphen{}win32}} driver using Zadig.
\end{sphinxadmonition}

\sphinxAtStartPar
This will launch the graphical interface of the Loadupch tool, allowing you to upload code to your CH552 microcontroller easily.

\end{enumerate}

\begin{sphinxadmonition}{tip}{Tip:}
\sphinxAtStartPar
If you need to uninstall the Loadupch tool for any reason, use the following command:

\begin{sphinxVerbatim}[commandchars=\\\{\}]
pip\PYG{+w}{ }uninstall\PYG{+w}{ }loadupch
\end{sphinxVerbatim}
\end{sphinxadmonition}

\begin{sphinxadmonition}{note}{Note:}
\sphinxAtStartPar
Requires the \sphinxtitleref{libusb\sphinxhyphen{}win32} driver to be installed using Zadig.
\end{sphinxadmonition}

\end{sphinxuseclass}
\sphinxstepscope


\chapter{General Board Control}
\label{\detokenize{generalboardcontrol:general-board-control}}\label{\detokenize{generalboardcontrol::doc}}
\sphinxAtStartPar
The CH552, characterized by its compact size, native USB connectivity, and 16 KB memory (with 14 KB usable), enables the creation of simple yet effective programs. This allows for greater control in implementing various applications. The choice of this microcontroller is based on its affordability, ease of connection, and compatibility with various operating systems.


\section{Recommended Operating Conditions}
\label{\detokenize{generalboardcontrol:recommended-operating-conditions}}

\begin{savenotes}\sphinxattablestart
\sphinxthistablewithglobalstyle
\centering
\sphinxcapstartof{table}
\sphinxthecaptionisattop
\sphinxcaption{Recommended Operating Conditions}\label{\detokenize{generalboardcontrol:id1}}
\sphinxaftertopcaption
\begin{tabular}[t]{\X{25}{100}\X{25}{100}\X{50}{100}}
\sphinxtoprule
\sphinxstyletheadfamily 
\sphinxAtStartPar
Symbol
&\sphinxstyletheadfamily 
\sphinxAtStartPar
Description
&\sphinxstyletheadfamily 
\sphinxAtStartPar
Range
\\
\sphinxmidrule
\sphinxtableatstartofbodyhook
\sphinxAtStartPar
VUSB
&
\sphinxAtStartPar
Voltage supply via USB
&
\sphinxAtStartPar
3.14 to 5.255 V
\\
\sphinxhline
\sphinxAtStartPar
VIn
&
\sphinxAtStartPar
Voltage supply from pins
&
\sphinxAtStartPar
2.7 to 5.5 V
\\
\sphinxhline
\sphinxAtStartPar
Top
&
\sphinxAtStartPar
Operating temperature
&
\sphinxAtStartPar
\sphinxhyphen{}40 to 85 °C
\\
\sphinxbottomrule
\end{tabular}
\sphinxtableafterendhook\par
\sphinxattableend\end{savenotes}


\section{Voltage Selector}
\label{\detokenize{generalboardcontrol:voltage-selector}}
\sphinxAtStartPar
The development board utilizes a clever voltage selector system consisting of three pins and a jumper switch. The configuration of these pins determines the operating voltage of the board. By connecting the central pin to the +5V pin via the jumper, the board operates at 5V. On the other hand, by connecting the central pin to the +3.3V pin, the APK2112K regulator is activated, powering the board at 3.3V. It is crucial to ensure that the jumper switch is in the correct position according to the desired voltage to avoid possible damage to modules, components, and the board itself.

\begin{figure}[htbp]
\centering
\capstart

\noindent\sphinxincludegraphics[width=0.400\linewidth]{{selector}.png}
\caption{USB Connector}\label{\detokenize{generalboardcontrol:id2}}\label{\detokenize{generalboardcontrol:selector}}\end{figure}


\section{JST Connectors}
\label{\detokenize{generalboardcontrol:jst-connectors}}
\sphinxAtStartPar
The board features two 1mm JST connectors, linked to different pins. The first connector directly connects to GPIO 3.0 and 3.1 of the microcontroller, while the second one is linked to pins 3.2 and 1.5. Both connectors operate in parallel to the selected power supply voltage via the jumper switch. These connectors are compatible with QWIIC, STEMMA QT, or similar pin distribution protocols. It is essential to verify that the selector voltage matches the system voltage to avoid circuit damage. Additionally, these connectors allow the board to be powered and offer functionalities such as PWM and serial communication.

\begin{figure}[htbp]
\centering
\capstart

\noindent\sphinxincludegraphics[width=0.500\linewidth]{{jst}.png}
\caption{JST Connectors}\label{\detokenize{generalboardcontrol:id3}}\label{\detokenize{generalboardcontrol:jst}}\end{figure}


\section{Built\sphinxhyphen{}In LEDs}
\label{\detokenize{generalboardcontrol:built-in-leds}}
\sphinxAtStartPar
The board features two LEDs directly linked to the microcontroller. The first one is connected to pin 3.4, while the second one is a Neopixel LED connected to pin 3.3. This Neopixel provides an output with two headers, one connected to the data output and the other to the board’s ground, allowing for the external connection of more LEDs. To use this output, simply connect the DOUT pin to the DIN pin of the next LED in the row. As for power supply, you can use the VCC pin, provided that the external LEDs can operate at this voltage. Otherwise, it will be necessary to power them using an external source.

\begin{figure}[htbp]
\centering
\capstart

\noindent\sphinxincludegraphics[width=0.500\linewidth]{{neopixel}.png}
\caption{Neopixel LED}\label{\detokenize{generalboardcontrol:id4}}\label{\detokenize{generalboardcontrol:neopixel}}\end{figure}

\sphinxstepscope


\chapter{Analog to Digital Converter (ADC)}
\label{\detokenize{adc:analog-to-digital-converter-adc}}\label{\detokenize{adc::doc}}
\sphinxAtStartPar
The CH552 has four ADC channels, which can be used to read analog values from sensors. The ADC channels are multiplexed with the GPIO pins, so you can use any GPIO pin as an ADC input.
the distribution of the ADC channels is as follows.


\begin{savenotes}\sphinxattablestart
\sphinxthistablewithglobalstyle
\centering
\sphinxcapstartof{table}
\sphinxthecaptionisattop
\sphinxcaption{Pin Mapping}\label{\detokenize{adc:id1}}
\sphinxaftertopcaption
\begin{tabular}[t]{\X{10}{30}\X{20}{30}}
\sphinxtoprule
\sphinxstyletheadfamily 
\sphinxAtStartPar
PIN
&\sphinxstyletheadfamily 
\sphinxAtStartPar
ADC Channel
\\
\sphinxmidrule
\sphinxtableatstartofbodyhook
\sphinxAtStartPar
A0
&
\sphinxAtStartPar
P1.1
\\
\sphinxhline
\sphinxAtStartPar
A1
&
\sphinxAtStartPar
P1.4
\\
\sphinxhline
\sphinxAtStartPar
A2
&
\sphinxAtStartPar
P1.5
\\
\sphinxhline
\sphinxAtStartPar
A3
&
\sphinxAtStartPar
P3.2
\\
\sphinxbottomrule
\end{tabular}
\sphinxtableafterendhook\par
\sphinxattableend\end{savenotes}

\sphinxAtStartPar
The ADC has a resolution of 8 bits, which means it can read values from 0 to 255. The ADC can be configured to read values from 0 to 5V, or from 0 to 3.3V, depending on the VCC voltage:

\begin{sphinxVerbatim}[commandchars=\\\{\}]
\PYG{c+c1}{\PYGZsh{}include \PYGZdq{}src/config.h\PYGZdq{} // user configurations}
\PYG{c+c1}{\PYGZsh{}include \PYGZdq{}src/system.h\PYGZdq{} // system functions}
\PYG{c+c1}{\PYGZsh{}include \PYGZdq{}src/gpio.h\PYGZdq{}   // for GPIO}
\PYG{c+c1}{\PYGZsh{}include \PYGZdq{}src/delay.h\PYGZdq{}  // for delays}

\PYG{n}{void} \PYG{n}{main}\PYG{p}{(}\PYG{n}{void}\PYG{p}{)}
\PYG{p}{\PYGZob{}}
    \PYG{n}{CLK\PYGZus{}config}\PYG{p}{(}\PYG{p}{)}\PYG{p}{;}
    \PYG{n}{DLY\PYGZus{}ms}\PYG{p}{(}\PYG{l+m+mi}{5}\PYG{p}{)}\PYG{p}{;}

    \PYG{n}{ADC\PYGZus{}input}\PYG{p}{(}\PYG{n}{PIN\PYGZus{}ADC}\PYG{p}{)}\PYG{p}{;}

    \PYG{n}{ADC\PYGZus{}enable}\PYG{p}{(}\PYG{p}{)}\PYG{p}{;}
    \PYG{k}{while} \PYG{p}{(}\PYG{l+m+mi}{1}\PYG{p}{)}
    \PYG{p}{\PYGZob{}}
        \PYG{n+nb}{int} \PYG{n}{data} \PYG{o}{=} \PYG{n}{ADC\PYGZus{}read}\PYG{p}{(}\PYG{p}{)}\PYG{p}{;} \PYG{o}{/}\PYG{o}{/} \PYG{n}{Assuming} \PYG{n}{ADC\PYGZus{}read}\PYG{p}{(}\PYG{p}{)} \PYG{n}{returns} \PYG{n}{an} \PYG{n+nb}{int}
    \PYG{p}{\PYGZcb{}}
\PYG{p}{\PYGZcb{}}
\end{sphinxVerbatim}

\sphinxstepscope


\chapter{I2C (Inter\sphinxhyphen{}Integrated Circuit)}
\label{\detokenize{i2c:i2c-inter-integrated-circuit}}\label{\detokenize{i2c::doc}}
\sphinxAtStartPar
I2C is a serial communication protocol, so data is transferred bit by bit along a single wire (the SDA line). The SCL line is used to synchronize the data transfer. The I2C protocol is a master\sphinxhyphen{}slave protocol, which means that the communication is always initiated by the master device. A master device can communicate with one or multiple slave devices.

\begin{sphinxadmonition}{note}{Note:}
\sphinxAtStartPar
The microcontroller use bit banging to communicate with the I2C devices.
\end{sphinxadmonition}

\sphinxAtStartPar
All pin configurations are defined in the \sphinxtitleref{config.h} file. The I2C pins are defined as follows


\begin{savenotes}\sphinxattablestart
\sphinxthistablewithglobalstyle
\centering
\sphinxcapstartof{table}
\sphinxthecaptionisattop
\sphinxcaption{I2C Pinout}\label{\detokenize{i2c:id1}}
\sphinxaftertopcaption
\begin{tabular}[t]{\X{20}{60}\X{20}{60}\X{20}{60}}
\sphinxtoprule
\sphinxstyletheadfamily 
\sphinxAtStartPar
PIN
&\sphinxstyletheadfamily 
\sphinxAtStartPar
I2C 1
&\sphinxstyletheadfamily 
\sphinxAtStartPar
I2C 2
\\
\sphinxmidrule
\sphinxtableatstartofbodyhook
\sphinxAtStartPar
SDA
&
\sphinxAtStartPar
P15
&
\sphinxAtStartPar
P31
\\
\sphinxhline
\sphinxAtStartPar
SCLK
&
\sphinxAtStartPar
P32
&
\sphinxAtStartPar
P30
\\
\sphinxbottomrule
\end{tabular}
\sphinxtableafterendhook\par
\sphinxattableend\end{savenotes}


\section{SSD1306 Display}
\label{\detokenize{i2c:ssd1306-display}}
\sphinxAtStartPar
The OLED display is a 128x64 pixel monochrome display that uses the I2C protocol for communication.

\begin{figure}[htbp]
\centering
\capstart

\noindent\sphinxincludegraphics[width=0.400\linewidth]{{oled}.jpg}
\caption{SSD1306 Display}\label{\detokenize{i2c:id2}}\label{\detokenize{i2c:figura-ssd1306-display}}\end{figure}

\sphinxAtStartPar
The following code snippet shows how to initialize the OLED display and print a message on the screen:

\begin{sphinxVerbatim}[commandchars=\\\{\}]
\PYG{n}{void} \PYG{n}{beep}\PYG{p}{(}\PYG{n}{void}\PYG{p}{)} \PYG{p}{\PYGZob{}}
\PYG{n}{uint8\PYGZus{}t} \PYG{n}{i}\PYG{p}{;}
\PYG{k}{for}\PYG{p}{(}\PYG{n}{i}\PYG{o}{=}\PYG{l+m+mi}{255}\PYG{p}{;} \PYG{n}{i}\PYG{p}{;} \PYG{n}{i}\PYG{o}{\PYGZhy{}}\PYG{o}{\PYGZhy{}}\PYG{p}{)} \PYG{p}{\PYGZob{}}
    \PYG{n}{PIN\PYGZus{}low}\PYG{p}{(}\PYG{n}{PIN\PYGZus{}BUZZER}\PYG{p}{)}\PYG{p}{;}
    \PYG{n}{DLY\PYGZus{}us}\PYG{p}{(}\PYG{l+m+mi}{125}\PYG{p}{)}\PYG{p}{;}
    \PYG{n}{PIN\PYGZus{}high}\PYG{p}{(}\PYG{n}{PIN\PYGZus{}BUZZER}\PYG{p}{)}\PYG{p}{;}
    \PYG{n}{DLY\PYGZus{}us}\PYG{p}{(}\PYG{l+m+mi}{125}\PYG{p}{)}\PYG{p}{;}
\PYG{p}{\PYGZcb{}}
\PYG{p}{\PYGZcb{}}

\PYG{n}{void} \PYG{n}{main}\PYG{p}{(}\PYG{n}{void}\PYG{p}{)} \PYG{p}{\PYGZob{}}
\PYG{o}{/}\PYG{o}{/} \PYG{n}{Setup}
\PYG{n}{CLK\PYGZus{}config}\PYG{p}{(}\PYG{p}{)}\PYG{p}{;}                           \PYG{o}{/}\PYG{o}{/} \PYG{n}{configure} \PYG{n}{system} \PYG{n}{clock}
\PYG{n}{DLY\PYGZus{}ms}\PYG{p}{(}\PYG{l+m+mi}{5}\PYG{p}{)}\PYG{p}{;}                              \PYG{o}{/}\PYG{o}{/} \PYG{n}{wait} \PYG{k}{for} \PYG{n}{clock} \PYG{n}{to} \PYG{n}{stabilize}

\PYG{n}{OLED\PYGZus{}init}\PYG{p}{(}\PYG{p}{)}\PYG{p}{;}                            \PYG{o}{/}\PYG{o}{/} \PYG{n}{init} \PYG{n}{OLED}

\PYG{n}{OLED\PYGZus{}print}\PYG{p}{(}\PYG{l+s+s2}{\PYGZdq{}}\PYG{l+s+s2}{*  UNITelectronics  *}\PYG{l+s+s2}{\PYGZdq{}}\PYG{p}{)}\PYG{p}{;}
\PYG{n}{OLED\PYGZus{}print}\PYG{p}{(}\PYG{l+s+s2}{\PYGZdq{}}\PYG{l+s+s2}{\PYGZhy{}\PYGZhy{}\PYGZhy{}\PYGZhy{}\PYGZhy{}\PYGZhy{}\PYGZhy{}\PYGZhy{}\PYGZhy{}\PYGZhy{}\PYGZhy{}\PYGZhy{}\PYGZhy{}\PYGZhy{}\PYGZhy{}\PYGZhy{}\PYGZhy{}\PYGZhy{}\PYGZhy{}\PYGZhy{}\PYGZhy{}}\PYG{l+s+se}{\PYGZbs{}n}\PYG{l+s+s2}{\PYGZdq{}}\PYG{p}{)}\PYG{p}{;}
\PYG{n}{OLED\PYGZus{}print}\PYG{p}{(}\PYG{l+s+s2}{\PYGZdq{}}\PYG{l+s+s2}{Ready}\PYG{l+s+se}{\PYGZbs{}n}\PYG{l+s+s2}{\PYGZdq{}}\PYG{p}{)}\PYG{p}{;}
\PYG{n}{beep}\PYG{p}{(}\PYG{p}{)}\PYG{p}{;}
\PYG{k}{while}\PYG{p}{(}\PYG{l+m+mi}{1}\PYG{p}{)} \PYG{p}{\PYGZob{}}

\PYG{p}{\PYGZcb{}}
\PYG{p}{\PYGZcb{}}
\end{sphinxVerbatim}

\begin{sphinxadmonition}{tip}{Tip:}
\sphinxAtStartPar
Remember to includes library corresponding to the OLED display in the file.
\end{sphinxadmonition}

\sphinxstepscope


\chapter{Interrupts}
\label{\detokenize{interrupts:interrupts}}\label{\detokenize{interrupts::doc}}
\sphinxAtStartPar
The CH552 microcontroller supports 14 sets of interrupt signal sources. These include 6 sets of interrupts
(INT0, T0, INT1, T1, UART0, and T2), which are compatible with the standard MCS51, and 8 sets of extended
interrupts (SPI0, TKEY, USB, ADC, UART1, PWMX, GPIO, and WDOG). The GPIO interrupt can be chosen from 7 I/O
pins.


\begin{savenotes}\sphinxattablestart
\sphinxthistablewithglobalstyle
\centering
\sphinxcapstartof{table}
\sphinxthecaptionisattop
\sphinxcaption{Default priority sequence of interrupt sources}\label{\detokenize{interrupts:id1}}
\sphinxaftertopcaption
\begin{tabular}[t]{\X{20}{100}\X{20}{100}\X{10}{100}\X{30}{100}\X{20}{100}}
\sphinxtoprule
\sphinxstyletheadfamily 
\sphinxAtStartPar
Interrupt src.
&\sphinxstyletheadfamily 
\sphinxAtStartPar
Entry addr.
&\sphinxstyletheadfamily 
\sphinxAtStartPar
Interrupt \#
&\sphinxstyletheadfamily 
\sphinxAtStartPar
Description
&\sphinxstyletheadfamily 
\sphinxAtStartPar
Default prio. seq.
\\
\sphinxmidrule
\sphinxtableatstartofbodyhook
\sphinxAtStartPar
INT\_NO\_INT0
&
\sphinxAtStartPar
0x0003
&
\sphinxAtStartPar
0
&
\sphinxAtStartPar
External interrupt 0
&
\sphinxAtStartPar
High priority
\\
\sphinxhline
\sphinxAtStartPar
INT\_NO\_TMR0
&
\sphinxAtStartPar
0x000B
&
\sphinxAtStartPar
1
&
\sphinxAtStartPar
Timer 0 interrupt
&
\sphinxAtStartPar
↓
\\
\sphinxhline
\sphinxAtStartPar
INT\_NO\_INT1
&
\sphinxAtStartPar
0x0013
&
\sphinxAtStartPar
2
&
\sphinxAtStartPar
External interrupt 1
&
\sphinxAtStartPar
↓
\\
\sphinxhline
\sphinxAtStartPar
INT\_NO\_TMR1
&
\sphinxAtStartPar
0x001B
&
\sphinxAtStartPar
3
&
\sphinxAtStartPar
Timer 1 interrupt
&
\sphinxAtStartPar
↓
\\
\sphinxhline
\sphinxAtStartPar
INT\_NO\_UART0
&
\sphinxAtStartPar
0x0023
&
\sphinxAtStartPar
4
&
\sphinxAtStartPar
UART0 interrupt
&
\sphinxAtStartPar
↓
\\
\sphinxhline
\sphinxAtStartPar
INT\_NO\_TMR2
&
\sphinxAtStartPar
0x002B
&
\sphinxAtStartPar
5
&
\sphinxAtStartPar
Timer 2 interrupt
&
\sphinxAtStartPar
↓
\\
\sphinxhline
\sphinxAtStartPar
INT\_NO\_SPI0
&
\sphinxAtStartPar
0x0033
&
\sphinxAtStartPar
6
&
\sphinxAtStartPar
SPI0 interrupt
&
\sphinxAtStartPar
↓
\\
\sphinxhline
\sphinxAtStartPar
INT\_NO\_TKEY
&
\sphinxAtStartPar
0x003B
&
\sphinxAtStartPar
7
&
\sphinxAtStartPar
Touch key timer interrupt
&
\sphinxAtStartPar
↓
\\
\sphinxhline
\sphinxAtStartPar
INT\_NO\_USB
&
\sphinxAtStartPar
0x0043
&
\sphinxAtStartPar
8
&
\sphinxAtStartPar
USB interrupt
&
\sphinxAtStartPar
↓
\\
\sphinxhline
\sphinxAtStartPar
INT\_NO\_ADC
&
\sphinxAtStartPar
0x004B
&
\sphinxAtStartPar
9
&
\sphinxAtStartPar
ADC interrupt
&
\sphinxAtStartPar
↓
\\
\sphinxhline
\sphinxAtStartPar
INT\_NO\_UART1
&
\sphinxAtStartPar
0x0053
&
\sphinxAtStartPar
10
&
\sphinxAtStartPar
UART1 interrupt
&
\sphinxAtStartPar
↓
\\
\sphinxhline
\sphinxAtStartPar
INT\_NO\_PWMX
&
\sphinxAtStartPar
0x005B
&
\sphinxAtStartPar
11
&
\sphinxAtStartPar
PWM1/PWM2 interrupt
&
\sphinxAtStartPar
↓
\\
\sphinxhline
\sphinxAtStartPar
INT\_NO\_GPIO
&
\sphinxAtStartPar
0x0063
&
\sphinxAtStartPar
12
&
\sphinxAtStartPar
GPIO Interrupt
&
\sphinxAtStartPar
↓
\\
\sphinxhline
\sphinxAtStartPar
INT\_NO\_WDOG
&
\sphinxAtStartPar
0x006B
&
\sphinxAtStartPar
13
&
\sphinxAtStartPar
Watchdog timer interrupt
&
\sphinxAtStartPar
Low priority
\\
\sphinxbottomrule
\end{tabular}
\sphinxtableafterendhook\par
\sphinxattableend\end{savenotes}

\sphinxAtStartPar
The interrupt priority is determined by the interrupt number.


\section{Timer 0/1  Interrupts}
\label{\detokenize{interrupts:timer-0-1-interrupts}}
\sphinxAtStartPar
Timer0 and Timer1 are 16\sphinxhyphen{}bit timers/counters controlled by TCON and TMOD. TCON is responsible for timer/counter
T0 and T1 startup control, overflow interrupt, and external interrupt control. Each timer consists of dual
8\sphinxhyphen{}bit registers forming a 16\sphinxhyphen{}bit timing unit. Timer 0’s high byte counter is TH0, and its low byte counter
is TL0. Similarly, Timer 1’s high byte counter is TH1, and its low byte counter is TL1. Timer 1 can also
serve as the baud rate generator for UART0.
code example:

\begin{sphinxVerbatim}[commandchars=\\\{\}]
\PYG{n}{void} \PYG{n}{timer0\PYGZus{}interrupt}\PYG{p}{(}\PYG{n}{void}\PYG{p}{)} \PYG{n}{\PYGZus{}\PYGZus{}interrupt}\PYG{p}{(}\PYG{n}{INT\PYGZus{}NO\PYGZus{}TMR0}\PYG{p}{)}        \PYG{o}{/}\PYG{o}{*} \PYG{n}{Timer0} \PYG{n}{interrupt} \PYG{n}{service} \PYG{n}{routine} \PYG{p}{(}\PYG{n}{ISR}\PYG{p}{)} \PYG{o}{*}\PYG{o}{/}
\PYG{p}{\PYGZob{}}
    \PYG{n}{PIN\PYGZus{}toggle}\PYG{p}{(}\PYG{n}{PIN\PYGZus{}BUZZER}\PYG{p}{)}\PYG{p}{;}
    \PYG{n}{TH0} \PYG{o}{=} \PYG{l+m+mh}{0xFF}\PYG{p}{;}             \PYG{o}{/}\PYG{o}{*} \PYG{l+m+mi}{50}\PYG{n}{ms} \PYG{n}{timer} \PYG{n}{value} \PYG{o}{*}\PYG{o}{/}
    \PYG{n}{TL0} \PYG{o}{=} \PYG{l+m+mh}{0x00}\PYG{p}{;}
\PYG{p}{\PYGZcb{}}


\PYG{n+nb}{int} \PYG{n}{main}\PYG{p}{(}\PYG{n}{void}\PYG{p}{)}
\PYG{p}{\PYGZob{}}
    \PYG{n}{CLK\PYGZus{}config}\PYG{p}{(}\PYG{p}{)}\PYG{p}{;}
    \PYG{n}{DLY\PYGZus{}ms}\PYG{p}{(}\PYG{l+m+mi}{5}\PYG{p}{)}\PYG{p}{;}
    \PYG{n}{PIN\PYGZus{}output}\PYG{p}{(}\PYG{n}{PIN\PYGZus{}BUZZER}\PYG{p}{)}\PYG{p}{;}
    \PYG{n}{EA}  \PYG{o}{=} \PYG{l+m+mi}{1}\PYG{p}{;}                \PYG{o}{/}\PYG{o}{*} \PYG{n}{Enable} \PYG{k}{global} \PYG{n}{interrupt} \PYG{o}{*}\PYG{o}{/}
    \PYG{n}{ET0} \PYG{o}{=} \PYG{l+m+mi}{1}\PYG{p}{;}                \PYG{o}{/}\PYG{o}{*} \PYG{n}{Enable} \PYG{n}{timer0} \PYG{n}{interrupt} \PYG{o}{*}\PYG{o}{/}

    \PYG{n}{TH0} \PYG{o}{=} \PYG{l+m+mh}{0xFF}\PYG{p}{;}             \PYG{o}{/}\PYG{o}{*} \PYG{l+m+mi}{50}\PYG{n}{ms} \PYG{n}{timer} \PYG{n}{value} \PYG{o}{*}\PYG{o}{/}
    \PYG{n}{TL0} \PYG{o}{=} \PYG{l+m+mh}{0x00}\PYG{p}{;}
    \PYG{n}{TMOD} \PYG{o}{=} \PYG{l+m+mh}{0x01}\PYG{p}{;}            \PYG{o}{/}\PYG{o}{*} \PYG{n}{Timer0} \PYG{n}{mode1} \PYG{o}{*}\PYG{o}{/}
    \PYG{n}{TR0} \PYG{o}{=} \PYG{l+m+mi}{1}\PYG{p}{;}                \PYG{o}{/}\PYG{o}{*} \PYG{n}{Start} \PYG{n}{timer0} \PYG{o}{*}\PYG{o}{/}
    \PYG{k}{while}\PYG{p}{(}\PYG{l+m+mi}{1}\PYG{p}{)}\PYG{p}{;}
\PYG{p}{\PYGZcb{}}
\end{sphinxVerbatim}


\section{External Interrupts}
\label{\detokenize{interrupts:external-interrupts}}
\sphinxAtStartPar
INT0 and INT1 are external interrupt input pins. When an external interrupt occurs, the corresponding
interrupt service routine is executed. The external interrupt can be triggered by the falling edge,
rising edge, or both edges of the external interrupt input signal. The trigger mode is determined by the
external interrupt input pin.

\sphinxAtStartPar
code example:

\begin{sphinxVerbatim}[commandchars=\\\{\}]
\PYG{n}{void} \PYG{n}{ext0\PYGZus{}interrupt}\PYG{p}{(}\PYG{n}{void}\PYG{p}{)} \PYG{n}{\PYGZus{}\PYGZus{}interrupt}\PYG{p}{(}\PYG{n}{INT\PYGZus{}NO\PYGZus{}INT0}\PYG{p}{)}
\PYG{p}{\PYGZob{}}
    \PYG{n}{PIN\PYGZus{}toggle}\PYG{p}{(}\PYG{n}{PIN\PYGZus{}LED}\PYG{p}{)}\PYG{p}{;}
\PYG{p}{\PYGZcb{}}

\PYG{n+nb}{int} \PYG{n}{main}\PYG{p}{(}\PYG{n}{void}\PYG{p}{)}
\PYG{p}{\PYGZob{}}
    \PYG{n}{CLK\PYGZus{}config}\PYG{p}{(}\PYG{p}{)}\PYG{p}{;}
    \PYG{n}{DLY\PYGZus{}ms}\PYG{p}{(}\PYG{l+m+mi}{5}\PYG{p}{)}\PYG{p}{;}
    \PYG{n}{PIN\PYGZus{}output\PYGZus{}OD}\PYG{p}{(}\PYG{n}{PIN\PYGZus{}INT}\PYG{p}{)}\PYG{p}{;}
    \PYG{n}{PIN\PYGZus{}output}\PYG{p}{(}\PYG{n}{PIN\PYGZus{}LED}\PYG{p}{)}\PYG{p}{;}

    \PYG{n}{EA}  \PYG{o}{=} \PYG{l+m+mi}{1}\PYG{p}{;}     \PYG{o}{/}\PYG{o}{*} \PYG{n}{Enable} \PYG{k}{global} \PYG{n}{interrupt} \PYG{o}{*}\PYG{o}{/}
    \PYG{n}{EX0} \PYG{o}{=} \PYG{l+m+mi}{1}\PYG{p}{;}    \PYG{o}{/}\PYG{o}{/} \PYG{n}{Enable} \PYG{n}{INT0}
    \PYG{n}{IT0} \PYG{o}{=} \PYG{l+m+mi}{1}\PYG{p}{;}    \PYG{o}{/}\PYG{o}{/} \PYG{n}{INT0} \PYG{o+ow}{is} \PYG{n}{edge} \PYG{n}{triggered}

    \PYG{k}{while}\PYG{p}{(}\PYG{l+m+mi}{1}\PYG{p}{)}
    \PYG{p}{\PYGZob{}}
        \PYG{o}{/}\PYG{o}{/} \PYG{n}{Do} \PYG{n}{nothing}
    \PYG{p}{\PYGZcb{}}

\PYG{p}{\PYGZcb{}}
\end{sphinxVerbatim}

\sphinxstepscope


\chapter{PWM (Pulse Width Modulation)}
\label{\detokenize{pwm:pwm-pulse-width-modulation}}\label{\detokenize{pwm::doc}}
\sphinxAtStartPar
The PWM module is used to generate a PWM signal on a pin. The PWM signal is generated by changing the duty cycle of the signal. The duty cycle is the ratio of the time the signal is high to the total time of the signal. The PWM module can be used to control the brightness of an LED, the speed of a motor, or the position of a servo motor.

\sphinxAtStartPar
The board contain two PWM pins, which are PIN\_PWM and PIN\_PWM2. The PWM module can be used to generate a PWM signal on these pins:

\begin{sphinxVerbatim}[commandchars=\\\{\}]
\PYG{n}{PWM} \PYG{l+m+mi}{1} \PYG{p}{:} \PYG{n}{P30}\PYG{o}{/}\PYG{n}{P15}
\PYG{n}{PWM} \PYG{l+m+mi}{2} \PYG{p}{:} \PYG{n}{P31}\PYG{o}{/}\PYG{n}{P34}
\end{sphinxVerbatim}

\sphinxAtStartPar
Some of the functions provided by the PWM module are:

\begin{sphinxVerbatim}[commandchars=\\\{\}]
\PYG{c+c1}{\PYGZsh{}define MIN\PYGZus{}COUNTER 10}
\PYG{c+c1}{\PYGZsh{}define MAX\PYGZus{}COUNTER 254}
\PYG{c+c1}{\PYGZsh{}define STEP\PYGZus{}SIZE   10}

\PYG{n}{void} \PYG{n}{change\PYGZus{}pwm}\PYG{p}{(}\PYG{n+nb}{int} \PYG{n}{hex\PYGZus{}value}\PYG{p}{)}
\PYG{p}{\PYGZob{}}
    \PYG{n}{PWM\PYGZus{}write}\PYG{p}{(}\PYG{n}{PIN\PYGZus{}PWM}\PYG{p}{,} \PYG{n}{hex\PYGZus{}value}\PYG{p}{)}\PYG{p}{;}
\PYG{p}{\PYGZcb{}}
\PYG{n}{void} \PYG{n}{main}\PYG{p}{(}\PYG{n}{void}\PYG{p}{)}
\PYG{p}{\PYGZob{}}
    \PYG{n}{CLK\PYGZus{}config}\PYG{p}{(}\PYG{p}{)}\PYG{p}{;}
    \PYG{n}{DLY\PYGZus{}ms}\PYG{p}{(}\PYG{l+m+mi}{5}\PYG{p}{)}\PYG{p}{;}
    \PYG{n}{PWM\PYGZus{}set\PYGZus{}freq}\PYG{p}{(}\PYG{l+m+mi}{1}\PYG{p}{)}\PYG{p}{;}
    \PYG{n}{PIN\PYGZus{}output}\PYG{p}{(}\PYG{n}{PIN\PYGZus{}PWM}\PYG{p}{)}\PYG{p}{;}
    \PYG{n}{PWM\PYGZus{}start}\PYG{p}{(}\PYG{n}{PIN\PYGZus{}PWM}\PYG{p}{)}\PYG{p}{;}
    \PYG{n}{PWM\PYGZus{}write}\PYG{p}{(}\PYG{n}{PIN\PYGZus{}PWM}\PYG{p}{,} \PYG{l+m+mi}{0}\PYG{p}{)}\PYG{p}{;}
\PYG{k}{while} \PYG{p}{(}\PYG{l+m+mi}{1}\PYG{p}{)}
\PYG{p}{\PYGZob{}}
    \PYG{k}{for} \PYG{p}{(}\PYG{n+nb}{int} \PYG{n}{i} \PYG{o}{=} \PYG{n}{MIN\PYGZus{}COUNTER}\PYG{p}{;} \PYG{n}{i} \PYG{o}{\PYGZlt{}} \PYG{n}{MAX\PYGZus{}COUNTER}\PYG{p}{;} \PYG{n}{i}\PYG{o}{+}\PYG{o}{=}\PYG{n}{STEP\PYGZus{}SIZE}\PYG{p}{)}
    \PYG{p}{\PYGZob{}}
        \PYG{n}{change\PYGZus{}pwm}\PYG{p}{(}\PYG{n}{i}\PYG{p}{)}\PYG{p}{;}
        \PYG{n}{DLY\PYGZus{}ms}\PYG{p}{(}\PYG{l+m+mi}{20}\PYG{p}{)}\PYG{p}{;}
    \PYG{p}{\PYGZcb{}}
    \PYG{k}{for} \PYG{p}{(}\PYG{n+nb}{int} \PYG{n}{i} \PYG{o}{=} \PYG{n}{MAX\PYGZus{}COUNTER}\PYG{p}{;} \PYG{n}{i} \PYG{o}{\PYGZgt{}} \PYG{n}{MIN\PYGZus{}COUNTER}\PYG{p}{;} \PYG{n}{i}\PYG{o}{\PYGZhy{}}\PYG{o}{=}\PYG{n}{STEP\PYGZus{}SIZE}\PYG{p}{)}
    \PYG{p}{\PYGZob{}}
        \PYG{n}{change\PYGZus{}pwm}\PYG{p}{(}\PYG{n}{i}\PYG{p}{)}\PYG{p}{;}
        \PYG{n}{DLY\PYGZus{}ms}\PYG{p}{(}\PYG{l+m+mi}{20}\PYG{p}{)}\PYG{p}{;}
    \PYG{p}{\PYGZcb{}}
\PYG{p}{\PYGZcb{}}
\PYG{p}{\PYGZcb{}}
\end{sphinxVerbatim}

\sphinxstepscope


\chapter{WS2812}
\label{\detokenize{ws2812:ws2812}}\label{\detokenize{ws2812::doc}}
\sphinxAtStartPar
This is a simple library for controlling WS2812 LEDs with an CH552 microcontroller. It is based on the
\sphinxhref{https://cdn-shop.adafruit.com/datasheets/WS2812.pdf}{WS2812 datasheet} and the
\sphinxhref{https://www.waveshare.com/w/upload/1/1d/CH552T-DS-EN-1.0.pdf}{CH552 datasheet}.

\begin{figure}[htbp]
\centering
\capstart

\noindent\sphinxincludegraphics[width=0.400\linewidth]{{WS1280_LED}.jpg}
\caption{WS2812 LED Strip}\label{\detokenize{ws2812:id1}}\label{\detokenize{ws2812:figura-dualmcu-one}}\end{figure}


\begin{savenotes}\sphinxattablestart
\sphinxthistablewithglobalstyle
\centering
\sphinxcapstartof{table}
\sphinxthecaptionisattop
\sphinxcaption{Pin Mapping for WS2812}\label{\detokenize{ws2812:id2}}
\sphinxaftertopcaption
\begin{tabular}[t]{\X{10}{20}\X{10}{20}}
\sphinxtoprule
\sphinxstyletheadfamily 
\sphinxAtStartPar
PIN
&\sphinxstyletheadfamily 
\sphinxAtStartPar
GPIO CH552
\\
\sphinxmidrule
\sphinxtableatstartofbodyhook
\sphinxAtStartPar
DOUT
&
\sphinxAtStartPar
P3.3
\\
\sphinxbottomrule
\end{tabular}
\sphinxtableafterendhook\par
\sphinxattableend\end{savenotes}

\begin{sphinxVerbatim}[commandchars=\\\{\}]
\PYG{c+cp}{\PYGZsh{}}\PYG{c+cp}{define delay 100}
\PYG{c+cp}{\PYGZsh{}}\PYG{c+cp}{define NeoPixel 16 }\PYG{c+c1}{// Number Neopixel conect}
\PYG{c+cp}{\PYGZsh{}}\PYG{c+cp}{define level 100 }\PYG{c+c1}{// Ilumination level 0 to 255}

\PYG{k+kt}{void}\PYG{+w}{ }\PYG{n+nf}{randomColorSequence}\PYG{p}{(}\PYG{k+kt}{void}\PYG{p}{)}\PYG{+w}{ }\PYG{p}{\PYGZob{}}

\PYG{k}{for}\PYG{p}{(}\PYG{k+kt}{int}\PYG{+w}{ }\PYG{n}{j}\PYG{o}{=}\PYG{l+m+mi}{0}\PYG{p}{;}\PYG{n}{j}\PYG{o}{\PYGZlt{}}\PYG{n}{NeoPixel}\PYG{p}{;}\PYG{n}{j}\PYG{o}{+}\PYG{o}{+}\PYG{p}{)}\PYG{p}{\PYGZob{}}
\PYG{+w}{    }\PYG{k+kt}{uint8\PYGZus{}t}\PYG{+w}{ }\PYG{n}{red}\PYG{+w}{ }\PYG{o}{=}\PYG{+w}{ }\PYG{n}{rand}\PYG{p}{(}\PYG{p}{)}\PYG{+w}{ }\PYG{o}{\PYGZpc{}}\PYG{+w}{ }\PYG{n}{level}\PYG{p}{;}
\PYG{+w}{    }\PYG{k+kt}{uint8\PYGZus{}t}\PYG{+w}{ }\PYG{n}{green}\PYG{+w}{ }\PYG{o}{=}\PYG{+w}{ }\PYG{n}{rand}\PYG{p}{(}\PYG{p}{)}\PYG{+w}{ }\PYG{o}{\PYGZpc{}}\PYG{+w}{ }\PYG{n}{level}\PYG{p}{;}
\PYG{+w}{    }\PYG{k+kt}{uint8\PYGZus{}t}\PYG{+w}{ }\PYG{n}{blue}\PYG{+w}{ }\PYG{o}{=}\PYG{+w}{ }\PYG{n}{rand}\PYG{p}{(}\PYG{p}{)}\PYG{+w}{ }\PYG{o}{\PYGZpc{}}\PYG{+w}{ }\PYG{n}{level}\PYG{p}{;}
\PYG{+w}{    }\PYG{k+kt}{uint8\PYGZus{}t}\PYG{+w}{ }\PYG{n}{num}\PYG{+w}{ }\PYG{o}{=}\PYG{+w}{ }\PYG{n}{rand}\PYG{p}{(}\PYG{p}{)}\PYG{+w}{ }\PYG{o}{\PYGZpc{}}\PYG{+w}{ }\PYG{n}{NeoPixel}\PYG{p}{;}

\PYG{+w}{    }\PYG{k}{for}\PYG{p}{(}\PYG{k+kt}{int}\PYG{+w}{ }\PYG{n}{i}\PYG{o}{=}\PYG{l+m+mi}{0}\PYG{p}{;}\PYG{+w}{ }\PYG{n}{i}\PYG{o}{\PYGZlt{}}\PYG{n}{num}\PYG{p}{;}\PYG{+w}{ }\PYG{n}{i}\PYG{o}{+}\PYG{o}{+}\PYG{p}{)}\PYG{p}{\PYGZob{}}
\PYG{+w}{        }\PYG{n}{NEO\PYGZus{}writeColor}\PYG{p}{(}\PYG{l+m+mi}{0}\PYG{p}{,}\PYG{+w}{ }\PYG{l+m+mi}{0}\PYG{p}{,}\PYG{+w}{ }\PYG{l+m+mi}{0}\PYG{p}{)}\PYG{p}{;}
\PYG{+w}{    }\PYG{p}{\PYGZcb{}}
\PYG{+w}{    }\PYG{n}{NEO\PYGZus{}writeColor}\PYG{p}{(}\PYG{n}{red}\PYG{p}{,}\PYG{+w}{ }\PYG{n}{green}\PYG{p}{,}\PYG{+w}{ }\PYG{n}{blue}\PYG{p}{)}\PYG{p}{;}
\PYG{+w}{    }\PYG{n}{DLY\PYGZus{}ms}\PYG{p}{(}\PYG{n}{delay}\PYG{p}{)}\PYG{p}{;}
\PYG{+w}{    }\PYG{n}{NEO\PYGZus{}writeColor}\PYG{p}{(}\PYG{l+m+mi}{0}\PYG{p}{,}\PYG{+w}{ }\PYG{l+m+mi}{0}\PYG{p}{,}\PYG{+w}{ }\PYG{l+m+mi}{0}\PYG{p}{)}\PYG{p}{;}
\PYG{+w}{    }\PYG{p}{\PYGZcb{}}

\PYG{+w}{    }\PYG{k}{for}\PYG{p}{(}\PYG{k+kt}{int}\PYG{+w}{ }\PYG{n}{l}\PYG{o}{=}\PYG{l+m+mi}{0}\PYG{p}{;}\PYG{+w}{ }\PYG{n}{l}\PYG{o}{\PYGZlt{}}\PYG{l+m+mi}{9}\PYG{p}{;}\PYG{+w}{ }\PYG{n}{l}\PYG{o}{+}\PYG{o}{+}\PYG{p}{)}\PYG{p}{\PYGZob{}}
\PYG{+w}{        }\PYG{n}{NEO\PYGZus{}writeColor}\PYG{p}{(}\PYG{l+m+mi}{0}\PYG{p}{,}\PYG{+w}{ }\PYG{l+m+mi}{0}\PYG{p}{,}\PYG{+w}{ }\PYG{l+m+mi}{0}\PYG{p}{)}\PYG{p}{;}
\PYG{+w}{    }\PYG{p}{\PYGZcb{}}

\PYG{p}{\PYGZcb{}}

\PYG{k+kt}{void}\PYG{+w}{ }\PYG{n+nf}{colorSequence}\PYG{p}{(}\PYG{k+kt}{void}\PYG{p}{)}\PYG{+w}{ }\PYG{p}{\PYGZob{}}

\PYG{k}{for}\PYG{p}{(}\PYG{k+kt}{int}\PYG{+w}{ }\PYG{n}{j}\PYG{o}{=}\PYG{l+m+mi}{0}\PYG{p}{;}\PYG{n}{j}\PYG{o}{\PYGZlt{}}\PYG{o}{=}\PYG{n}{NeoPixel}\PYG{p}{;}\PYG{n}{j}\PYG{o}{+}\PYG{o}{+}\PYG{p}{)}\PYG{p}{\PYGZob{}}
\PYG{+w}{        }\PYG{k+kt}{uint8\PYGZus{}t}\PYG{+w}{ }\PYG{n}{red}\PYG{+w}{ }\PYG{o}{=}\PYG{+w}{ }\PYG{n}{rand}\PYG{p}{(}\PYG{p}{)}\PYG{+w}{ }\PYG{o}{\PYGZpc{}}\PYG{+w}{ }\PYG{n}{level}\PYG{p}{;}
\PYG{+w}{        }\PYG{k+kt}{uint8\PYGZus{}t}\PYG{+w}{ }\PYG{n}{green}\PYG{+w}{ }\PYG{o}{=}\PYG{+w}{ }\PYG{n}{rand}\PYG{p}{(}\PYG{p}{)}\PYG{+w}{ }\PYG{o}{\PYGZpc{}}\PYG{+w}{ }\PYG{n}{level}\PYG{p}{;}
\PYG{+w}{        }\PYG{k+kt}{uint8\PYGZus{}t}\PYG{+w}{ }\PYG{n}{blue}\PYG{+w}{ }\PYG{o}{=}\PYG{+w}{ }\PYG{n}{rand}\PYG{p}{(}\PYG{p}{)}\PYG{+w}{ }\PYG{o}{\PYGZpc{}}\PYG{+w}{ }\PYG{n}{level}\PYG{p}{;}
\PYG{+w}{    }\PYG{k}{for}\PYG{p}{(}\PYG{k+kt}{int}\PYG{+w}{ }\PYG{n}{i}\PYG{o}{=}\PYG{l+m+mi}{0}\PYG{p}{;}\PYG{+w}{ }\PYG{n}{i}\PYG{o}{\PYGZlt{}}\PYG{n}{j}\PYG{p}{;}\PYG{+w}{ }\PYG{n}{i}\PYG{o}{+}\PYG{o}{+}\PYG{p}{)}\PYG{p}{\PYGZob{}}
\PYG{+w}{        }\PYG{n}{NEO\PYGZus{}writeColor}\PYG{p}{(}\PYG{n}{red}\PYG{p}{,}\PYG{+w}{ }\PYG{n}{green}\PYG{p}{,}\PYG{+w}{ }\PYG{n}{blue}\PYG{p}{)}\PYG{p}{;}
\PYG{+w}{    }\PYG{p}{\PYGZcb{}}
\PYG{+w}{    }\PYG{n}{DLY\PYGZus{}ms}\PYG{p}{(}\PYG{n}{delay}\PYG{p}{)}\PYG{p}{;}
\PYG{+w}{    }\PYG{k}{for}\PYG{p}{(}\PYG{k+kt}{int}\PYG{+w}{ }\PYG{n}{l}\PYG{o}{=}\PYG{l+m+mi}{0}\PYG{p}{;}\PYG{+w}{ }\PYG{n}{l}\PYG{o}{\PYGZlt{}}\PYG{n}{j}\PYG{p}{;}\PYG{+w}{ }\PYG{n}{l}\PYG{o}{+}\PYG{o}{+}\PYG{p}{)}\PYG{p}{\PYGZob{}}
\PYG{+w}{        }\PYG{n}{NEO\PYGZus{}writeColor}\PYG{p}{(}\PYG{l+m+mi}{0}\PYG{p}{,}\PYG{+w}{ }\PYG{l+m+mi}{0}\PYG{p}{,}\PYG{+w}{ }\PYG{l+m+mi}{0}\PYG{p}{)}\PYG{p}{;}
\PYG{+w}{    }\PYG{p}{\PYGZcb{}}
\PYG{p}{\PYGZcb{}}
\PYG{p}{\PYGZcb{}}

\PYG{c+c1}{// ===================================================================================}
\PYG{c+c1}{// Main Function}
\PYG{c+c1}{// ===================================================================================}
\PYG{k+kt}{void}\PYG{+w}{ }\PYG{n+nf}{main}\PYG{p}{(}\PYG{k+kt}{void}\PYG{p}{)}\PYG{+w}{ }\PYG{p}{\PYGZob{}}
\PYG{n}{NEO\PYGZus{}init}\PYG{p}{(}\PYG{p}{)}\PYG{p}{;}\PYG{+w}{                       }\PYG{c+c1}{// init NeoPixels}
\PYG{n}{CLK\PYGZus{}config}\PYG{p}{(}\PYG{p}{)}\PYG{p}{;}\PYG{+w}{                     }\PYG{c+c1}{// configure system clock}
\PYG{n}{DLY\PYGZus{}ms}\PYG{p}{(}\PYG{n}{delay}\PYG{p}{)}\PYG{p}{;}\PYG{+w}{                       }\PYG{c+c1}{// wait for clock to settle}

\PYG{c+c1}{// Loop}
\PYG{k}{while}\PYG{+w}{ }\PYG{p}{(}\PYG{l+m+mi}{1}\PYG{p}{)}\PYG{+w}{ }\PYG{p}{\PYGZob{}}
\PYG{+w}{    }\PYG{n}{randomColorSequence}\PYG{p}{(}\PYG{p}{)}\PYG{p}{;}
\PYG{+w}{    }\PYG{n}{DLY\PYGZus{}ms}\PYG{p}{(}\PYG{l+m+mi}{100}\PYG{p}{)}\PYG{p}{;}
\PYG{+w}{    }\PYG{n}{colorSequence}\PYG{p}{(}\PYG{p}{)}\PYG{p}{;}
\PYG{+w}{    }\PYG{n}{DLY\PYGZus{}ms}\PYG{p}{(}\PYG{l+m+mi}{100}\PYG{p}{)}\PYG{p}{;}
\PYG{p}{\PYGZcb{}}
\PYG{p}{\PYGZcb{}}
\end{sphinxVerbatim}

\sphinxstepscope


\chapter{Communication Serial CDC}
\label{\detokenize{cdc:communication-serial-cdc}}\label{\detokenize{cdc::doc}}
\sphinxAtStartPar
Cocket Nova development board  is compatible with the USB CDC (Communication Device Class) protocol. This allows the Cocket Nova
to be used as a virtual serial port. The CDC protocol is supported by most operating systems, including
Windows, Linux, and macOS.

\begin{sphinxadmonition}{note}{Note:}
\sphinxAtStartPar
The CDC protocol is implemented using the USB peripheral of the microcontroller. The USB peripheral is
configured as a virtual serial port, which allows the microcontroller to communicate with the host computer
using the USB cable.
\end{sphinxadmonition}

\sphinxAtStartPar
The following code snippet shows how to configure the USB peripheral as a virtual serial port and this method
only recive information from the host

\begin{sphinxVerbatim}[commandchars=\\\{\}]
\PYG{k+kt}{void}\PYG{+w}{ }\PYG{n+nf}{main}\PYG{p}{(}\PYG{k+kt}{void}\PYG{p}{)}\PYG{+w}{ }\PYG{p}{\PYGZob{}}
\PYG{c+c1}{// Setup}
\PYG{n}{CLK\PYGZus{}config}\PYG{p}{(}\PYG{p}{)}\PYG{p}{;}\PYG{+w}{                           }\PYG{c+c1}{// configure system clock}
\PYG{n}{DLY\PYGZus{}ms}\PYG{p}{(}\PYG{l+m+mi}{5}\PYG{p}{)}\PYG{p}{;}\PYG{+w}{                              }\PYG{c+c1}{// wait for clock to stabilize}
\PYG{n}{CDC\PYGZus{}init}\PYG{p}{(}\PYG{p}{)}\PYG{p}{;}\PYG{+w}{                             }\PYG{c+c1}{// init USB CDC}

\PYG{c+c1}{// Loop}
\PYG{k}{while}\PYG{p}{(}\PYG{l+m+mi}{1}\PYG{p}{)}\PYG{+w}{ }\PYG{p}{\PYGZob{}}
\PYG{+w}{    }\PYG{k}{if}\PYG{p}{(}\PYG{n}{CDC\PYGZus{}available}\PYG{p}{(}\PYG{p}{)}\PYG{p}{)}\PYG{+w}{ }\PYG{p}{\PYGZob{}}\PYG{+w}{                 }\PYG{c+c1}{// something coming in?}
\PYG{+w}{    }\PYG{k+kt}{char}\PYG{+w}{ }\PYG{n}{c}\PYG{+w}{ }\PYG{o}{=}\PYG{+w}{ }\PYG{n}{CDC\PYGZus{}read}\PYG{p}{(}\PYG{p}{)}\PYG{p}{;}\PYG{+w}{                }\PYG{c+c1}{// read the character ...}
\PYG{+w}{    }\PYG{n}{CDC\PYGZus{}writeflush}\PYG{p}{(}\PYG{n}{c}\PYG{p}{)}\PYG{p}{;}\PYG{+w}{                  }\PYG{c+c1}{// ... and send it back}
\PYG{+w}{    }\PYG{p}{\PYGZcb{}}
\PYG{p}{\PYGZcb{}}
\PYG{p}{\PYGZcb{}}
\end{sphinxVerbatim}


\section{USB CDC Serial Configuration for CH55x Microcontrollers}
\label{\detokenize{cdc:usb-cdc-serial-configuration-for-ch55x-microcontrollers}}

\subsection{USB Passthrough for CH55x Microcontrollers}
\label{\detokenize{cdc:usb-passthrough-for-ch55x-microcontrollers}}
\sphinxAtStartPar
This project implements a simple USB passthrough functionality using CH551, CH552, or CH554 microcontrollers. The microcontroller acts as a USB Communication Device Class (CDC), enabling serial communication over USB. Data received via USB is immediately sent back to the host computer.
Wiring

\sphinxAtStartPar
Connect the CH55x development board to your PC via USB. It should be automatically detected as a CDC device.
Compilation Instructions:

\begin{sphinxVerbatim}[commandchars=\\\{\}]
\PYG{n}{Chip}\PYG{p}{:} \PYG{n}{CH551}\PYG{p}{,} \PYG{n}{CH552}\PYG{p}{,} \PYG{o+ow}{or} \PYG{n}{CH554}
\PYG{n}{Clock}\PYG{p}{:} \PYG{l+m+mi}{16} \PYG{n}{MHz} \PYG{n}{internal}
\PYG{n}{Adjust} \PYG{n}{firmware} \PYG{n}{parameters} \PYG{o+ow}{in} \PYG{n}{src}\PYG{o}{/}\PYG{n}{config}\PYG{o}{.}\PYG{n}{h} \PYG{k}{if} \PYG{n}{necessary}\PYG{o}{.}
\PYG{n}{Ensure} \PYG{n}{SDCC} \PYG{n}{toolchain} \PYG{o+ow}{and} \PYG{n}{Python3} \PYG{k}{with} \PYG{n}{PyUSB} \PYG{n}{are} \PYG{n}{installed}\PYG{o}{.}
\PYG{n}{Press} \PYG{n}{the} \PYG{n}{BOOT} \PYG{n}{button} \PYG{n}{on} \PYG{n}{the} \PYG{n}{board} \PYG{o+ow}{and} \PYG{n}{keep} \PYG{n}{it} \PYG{n}{pressed} \PYG{k}{while} \PYG{n}{connecting} \PYG{n}{it} \PYG{n}{via} \PYG{n}{USB} \PYG{n}{to} \PYG{n}{your} \PYG{n}{PC}\PYG{o}{.}
\PYG{n}{Run} \PYG{n}{make} \PYG{n}{flash} \PYG{n}{immediately} \PYG{n}{afterwards} \PYG{n}{to} \PYG{n}{flash} \PYG{n}{the} \PYG{n}{firmware}\PYG{o}{.}
\PYG{n}{For} \PYG{n}{compilation} \PYG{n}{using} \PYG{n}{Arduino} \PYG{n}{IDE}\PYG{p}{,} \PYG{n}{refer} \PYG{n}{to} \PYG{n}{instructions} \PYG{o+ow}{in} \PYG{n}{the} \PYG{o}{.}\PYG{n}{ino} \PYG{n}{file}\PYG{o}{.}
\end{sphinxVerbatim}


\subsection{USB CDC Serial to UART Bridge for CH55x Microcontrollers}
\label{\detokenize{cdc:usb-cdc-serial-to-uart-bridge-for-ch55x-microcontrollers}}
\sphinxAtStartPar
This project implements a USB CDC to UART bridge functionality using ch552. The microcontroller acts as a USB
Communication Device Class (CDC), allowing serial communication over USB. Data received via USB is sent to the
UART interface and vice versa.

\begin{sphinxVerbatim}[commandchars=\\\{\}]
\PYG{c+c1}{// Prototypes for used interrupts}
\PYG{k+kt}{void}\PYG{+w}{ }\PYG{n+nf}{USB\PYGZus{}interrupt}\PYG{p}{(}\PYG{k+kt}{void}\PYG{p}{)}\PYG{p}{;}
\PYG{k+kt}{void}\PYG{+w}{ }\PYG{n+nf}{USB\PYGZus{}ISR}\PYG{p}{(}\PYG{k+kt}{void}\PYG{p}{)}\PYG{+w}{ }\PYG{n}{\PYGZus{}\PYGZus{}interrupt}\PYG{p}{(}\PYG{n}{INT\PYGZus{}NO\PYGZus{}USB}\PYG{p}{)}\PYG{+w}{ }\PYG{p}{\PYGZob{}}
\PYG{n}{USB\PYGZus{}interrupt}\PYG{p}{(}\PYG{p}{)}\PYG{p}{;}
\PYG{p}{\PYGZcb{}}

\PYG{k+kt}{void}\PYG{+w}{ }\PYG{n+nf}{UART\PYGZus{}interrupt}\PYG{p}{(}\PYG{k+kt}{void}\PYG{p}{)}\PYG{p}{;}
\PYG{k+kt}{void}\PYG{+w}{ }\PYG{n+nf}{UART\PYGZus{}ISR}\PYG{p}{(}\PYG{k+kt}{void}\PYG{p}{)}\PYG{+w}{ }\PYG{n}{\PYGZus{}\PYGZus{}interrupt}\PYG{p}{(}\PYG{n}{INT\PYGZus{}NO\PYGZus{}UART0}\PYG{p}{)}\PYG{+w}{ }\PYG{p}{\PYGZob{}}
\PYG{n}{UART\PYGZus{}interrupt}\PYG{p}{(}\PYG{p}{)}\PYG{p}{;}
\PYG{p}{\PYGZcb{}}
\PYG{c+c1}{// ===================================================================================}
\PYG{c+c1}{// Main Function}
\PYG{c+c1}{// ===================================================================================}
\PYG{k+kt}{void}\PYG{+w}{ }\PYG{n+nf}{main}\PYG{p}{(}\PYG{k+kt}{void}\PYG{p}{)}\PYG{+w}{ }\PYG{p}{\PYGZob{}}
\PYG{c+c1}{// Setup}
\PYG{n}{CLK\PYGZus{}config}\PYG{p}{(}\PYG{p}{)}\PYG{p}{;}\PYG{+w}{                             }\PYG{c+c1}{// configure system clock}
\PYG{n}{DLY\PYGZus{}ms}\PYG{p}{(}\PYG{l+m+mi}{10}\PYG{p}{)}\PYG{p}{;}\PYG{+w}{                               }\PYG{c+c1}{// wait for clock to settle}
\PYG{n}{UART\PYGZus{}init}\PYG{p}{(}\PYG{p}{)}\PYG{p}{;}\PYG{+w}{                              }\PYG{c+c1}{// init UART}
\PYG{n}{CDC\PYGZus{}init}\PYG{p}{(}\PYG{p}{)}\PYG{p}{;}\PYG{+w}{                               }\PYG{c+c1}{// init virtual COM}

\PYG{c+c1}{// Loop}
\PYG{k}{while}\PYG{p}{(}\PYG{l+m+mi}{1}\PYG{p}{)}\PYG{+w}{ }\PYG{p}{\PYGZob{}}
\PYG{+w}{    }\PYG{c+c1}{// Handle virtual COM}
\PYG{+w}{    }\PYG{k}{if}\PYG{p}{(}\PYG{n}{CDC\PYGZus{}available}\PYG{p}{(}\PYG{p}{)}\PYG{+w}{ }\PYG{o}{\PYGZam{}}\PYG{o}{\PYGZam{}}\PYG{+w}{ }\PYG{n}{UART\PYGZus{}ready}\PYG{p}{(}\PYG{p}{)}\PYG{p}{)}\PYG{+w}{ }\PYG{n}{UART\PYGZus{}write}\PYG{p}{(}\PYG{n}{CDC\PYGZus{}read}\PYG{p}{(}\PYG{p}{)}\PYG{p}{)}\PYG{p}{;}
\PYG{+w}{    }\PYG{k}{if}\PYG{p}{(}\PYG{n}{UART\PYGZus{}available}\PYG{p}{(}\PYG{p}{)}\PYG{+w}{ }\PYG{o}{\PYGZam{}}\PYG{o}{\PYGZam{}}\PYG{+w}{ }\PYG{n}{CDC\PYGZus{}getDTR}\PYG{p}{(}\PYG{p}{)}\PYG{p}{)}\PYG{+w}{ }\PYG{p}{\PYGZob{}}
\PYG{+w}{    }\PYG{k}{while}\PYG{p}{(}\PYG{n}{UART\PYGZus{}available}\PYG{p}{(}\PYG{p}{)}\PYG{p}{)}\PYG{+w}{ }\PYG{n}{CDC\PYGZus{}write}\PYG{p}{(}\PYG{n}{UART\PYGZus{}read}\PYG{p}{(}\PYG{p}{)}\PYG{p}{)}\PYG{p}{;}
\PYG{+w}{    }\PYG{n}{CDC\PYGZus{}flush}\PYG{p}{(}\PYG{p}{)}\PYG{p}{;}
\PYG{+w}{    }\PYG{p}{\PYGZcb{}}
\PYG{p}{\PYGZcb{}}
\PYG{p}{\PYGZcb{}}
\end{sphinxVerbatim}


\subsection{USB CDC PWM Controller for CH55x Microcontrollers}
\label{\detokenize{cdc:usb-cdc-pwm-controller-for-ch55x-microcontrollers}}
\sphinxAtStartPar
This project implements a USB CDC controlled PWM functionality using CH551, CH552, or CH554 microcontrollers. The microcontroller acts as a USB Communication Device Class (CDC), allowing serial communication over USB. Data received via USB is used to set the PWM value (0\sphinxhyphen{}255).
Wiring

\sphinxAtStartPar
Connect the CH55x development board to your PC via USB. It should be automatically detected as a CDC device.
Compilation Instructions:

\begin{sphinxVerbatim}[commandchars=\\\{\}]
\PYG{n}{Chip}\PYG{p}{:} \PYG{n}{CH551}\PYG{p}{,} \PYG{n}{CH552}\PYG{p}{,} \PYG{o+ow}{or} \PYG{n}{CH554}
\PYG{n}{Clock}\PYG{p}{:} \PYG{l+m+mi}{16} \PYG{n}{MHz} \PYG{n}{internal}
\PYG{n}{Adjust} \PYG{n}{firmware} \PYG{n}{parameters} \PYG{o+ow}{in} \PYG{n}{src}\PYG{o}{/}\PYG{n}{config}\PYG{o}{.}\PYG{n}{h} \PYG{k}{if} \PYG{n}{necessary}\PYG{o}{.}
\PYG{n}{Ensure} \PYG{n}{SDCC} \PYG{n}{toolchain} \PYG{o+ow}{and} \PYG{n}{Python3} \PYG{k}{with} \PYG{n}{PyUSB} \PYG{n}{are} \PYG{n}{installed}\PYG{o}{.}
\PYG{n}{Press} \PYG{n}{the} \PYG{n}{BOOT} \PYG{n}{button} \PYG{n}{on} \PYG{n}{the} \PYG{n}{board} \PYG{o+ow}{and} \PYG{n}{keep} \PYG{n}{it} \PYG{n}{pressed} \PYG{k}{while} \PYG{n}{connecting} \PYG{n}{it} \PYG{n}{via} \PYG{n}{USB} \PYG{n}{to} \PYG{n}{your} \PYG{n}{PC}\PYG{o}{.}
\PYG{n}{Run} \PYG{n}{make} \PYG{n}{flash} \PYG{n}{immediately} \PYG{n}{afterwards} \PYG{n}{to} \PYG{n}{flash} \PYG{n}{the} \PYG{n}{firmware}\PYG{o}{.}
\PYG{n}{For} \PYG{n}{compilation} \PYG{n}{using} \PYG{n}{Arduino} \PYG{n}{IDE}\PYG{p}{,} \PYG{n}{refer} \PYG{n}{to} \PYG{n}{instructions} \PYG{o+ow}{in} \PYG{n}{the} \PYG{o}{.}\PYG{n}{ino} \PYG{n}{file}\PYG{o}{.}
\end{sphinxVerbatim}


\section{Linux Configuration for USB CDC Devices}
\label{\detokenize{cdc:linux-configuration-for-usb-cdc-devices}}
\sphinxAtStartPar
To configure permissions for USB CDC devices (/dev/ttyACM0), follow these steps:

\sphinxAtStartPar
Create a new udev rule file:

\begin{sphinxVerbatim}[commandchars=\\\{\}]
sudo\PYG{+w}{ }nano\PYG{+w}{ }/etc/udev/rules.d/99\PYGZhy{}custom\PYGZhy{}usb.rules
\end{sphinxVerbatim}

\sphinxAtStartPar
Add the following rule to the file (replace idVendor and idProduct with your device’s actual IDs):

\begin{sphinxVerbatim}[commandchars=\\\{\}]
\PYG{n+nv}{SUBSYSTEM}\PYG{o}{=}\PYG{o}{=}\PYG{l+s+s2}{\PYGZdq{}tty\PYGZdq{}},\PYG{+w}{ }ATTRS\PYG{o}{\PYGZob{}}idVendor\PYG{o}{\PYGZcb{}}\PYG{o}{=}\PYG{o}{=}\PYG{l+s+s2}{\PYGZdq{}1209\PYGZdq{}},\PYG{+w}{ }ATTRS\PYG{o}{\PYGZob{}}idProduct\PYG{o}{\PYGZcb{}}\PYG{o}{=}\PYG{o}{=}\PYG{l+s+s2}{\PYGZdq{}27dd\PYGZdq{}},\PYG{+w}{ }\PYG{n+nv}{GROUP}\PYG{o}{=}\PYG{l+s+s2}{\PYGZdq{}dialout\PYGZdq{}},\PYG{+w}{ }\PYG{n+nv}{MODE}\PYG{o}{=}\PYG{l+s+s2}{\PYGZdq{}0666\PYGZdq{}}
\end{sphinxVerbatim}

\sphinxAtStartPar
Save the file (Ctrl + O in nano, then Enter) and exit nano (Ctrl + X).

\sphinxAtStartPar
Reload udev rules for changes to take effect:

\begin{sphinxVerbatim}[commandchars=\\\{\}]
sudo\PYG{+w}{ }udevadm\PYG{+w}{ }control\PYG{+w}{ }\PYGZhy{}\PYGZhy{}reload\PYGZhy{}rules
\end{sphinxVerbatim}


\subsection{Example Commands for Serial Communication}
\label{\detokenize{cdc:example-commands-for-serial-communication}}
\sphinxAtStartPar
Send data to USB device:

\begin{sphinxVerbatim}[commandchars=\\\{\}]
\PYG{n+nb}{echo}\PYG{+w}{ }\PYGZhy{}e\PYG{+w}{ }\PYG{l+s+s1}{\PYGZsq{}Hello World!\PYGZbs{}n\PYGZsq{}}\PYG{+w}{ }\PYGZgt{}\PYG{+w}{ }/dev/ttyACM0
\end{sphinxVerbatim}

\sphinxAtStartPar
Read data from USB device:

\begin{sphinxVerbatim}[commandchars=\\\{\}]
cat\PYG{+w}{ }/dev/ttyACM0
\end{sphinxVerbatim}

\sphinxAtStartPar
These commands allow you to interact with USB CDC devices connected to your Linux system. Adjust the device path (/dev/ttyACM0) as per your setup.


\section{Windows Configuration for USB CDC Devices}
\label{\detokenize{cdc:windows-configuration-for-usb-cdc-devices}}
\sphinxAtStartPar
To configure permissions for USB CDC devices in Windows, follow these steps:
\begin{enumerate}
\sphinxsetlistlabels{\arabic}{enumi}{enumii}{}{.}%
\item {} 
\sphinxAtStartPar
Identify the device’s COM port number in Device Manager.

\end{enumerate}

\begin{figure}[htbp]
\centering
\capstart

\noindent\sphinxincludegraphics{{cdc_serial_vid_pid}.png}
\caption{CDC Serial Device Manager}\label{\detokenize{cdc:id1}}\label{\detokenize{cdc:figura-device-manager}}\end{figure}
\begin{enumerate}
\sphinxsetlistlabels{\arabic}{enumi}{enumii}{}{.}%
\setcounter{enumi}{1}
\item {} 
\sphinxAtStartPar
Right\sphinxhyphen{}click on the device and select Properties.

\item {} 
\sphinxAtStartPar
Open \sphinxhref{https://zadig.akeo.ie/}{Zadig}.

\item {} 
\sphinxAtStartPar
Go to Options \textgreater{} List All Devices.

\item {} 
\sphinxAtStartPar
Select the device from the drop\sphinxhyphen{}down list.

\end{enumerate}

\begin{figure}[htbp]
\centering
\capstart

\noindent\sphinxincludegraphics{{cdc_serial}.png}
\caption{Zadig CDC Device Selection}\label{\detokenize{cdc:id2}}\label{\detokenize{cdc:figura-zadig}}\end{figure}

\sphinxstepscope


\chapter{How generarate report of errors}
\label{\detokenize{report:how-generarate-report-of-errors}}\label{\detokenize{report::doc}}
\sphinxAtStartPar
This is a guide to generate report of errors.

\sphinxAtStartPar
Unit Electronics is a company that produces electronic devices. The company has a quality control
department that is responsible for checking the quality of the devices designed.

\sphinxAtStartPar
you can get the report of errors by following the steps below:
\begin{enumerate}
\sphinxsetlistlabels{\arabic}{enumi}{enumii}{}{.}%
\item {} 
\sphinxAtStartPar
Got to the Unit Electronics github repository.

\item {} 
\sphinxAtStartPar
Click on the issues tab.

\item {} 
\sphinxAtStartPar
Click on the new issue button.

\item {} 
\sphinxAtStartPar
Fill in the title and description of the issue.

\item {} 
\sphinxAtStartPar
Click on the submit button.

\end{enumerate}

\sphinxAtStartPar
The quality control department will review the issue and take the necessary action to resolve it.



\renewcommand{\indexname}{Index}
\printindex
\end{document}