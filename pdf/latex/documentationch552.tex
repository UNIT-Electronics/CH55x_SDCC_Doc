%% Generated by Sphinx.
\def\sphinxdocclass{report}
\documentclass[letterpaper,10pt,english]{sphinxmanual}
\ifdefined\pdfpxdimen
   \let\sphinxpxdimen\pdfpxdimen\else\newdimen\sphinxpxdimen
\fi \sphinxpxdimen=.75bp\relax
\ifdefined\pdfimageresolution
    \pdfimageresolution= \numexpr \dimexpr1in\relax/\sphinxpxdimen\relax
\fi
%% let collapsible pdf bookmarks panel have high depth per default
\PassOptionsToPackage{bookmarksdepth=5}{hyperref}

\PassOptionsToPackage{booktabs}{sphinx}
\PassOptionsToPackage{colorrows}{sphinx}

\PassOptionsToPackage{warn}{textcomp}
\usepackage[utf8]{inputenc}
\ifdefined\DeclareUnicodeCharacter
% support both utf8 and utf8x syntaxes
  \ifdefined\DeclareUnicodeCharacterAsOptional
    \def\sphinxDUC#1{\DeclareUnicodeCharacter{"#1}}
  \else
    \let\sphinxDUC\DeclareUnicodeCharacter
  \fi
  \sphinxDUC{00A0}{\nobreakspace}
  \sphinxDUC{2500}{\sphinxunichar{2500}}
  \sphinxDUC{2502}{\sphinxunichar{2502}}
  \sphinxDUC{2514}{\sphinxunichar{2514}}
  \sphinxDUC{251C}{\sphinxunichar{251C}}
  \sphinxDUC{2572}{\textbackslash}
\fi
\usepackage{cmap}
\usepackage[T1]{fontenc}
\usepackage{amsmath,amssymb,amstext}
\usepackage{babel}



\usepackage{tgtermes}
\usepackage{tgheros}
\renewcommand{\ttdefault}{txtt}



\usepackage[Bjarne]{fncychap}
\usepackage{sphinx}

\fvset{fontsize=auto}
\usepackage{geometry}


% Include hyperref last.
\usepackage{hyperref}
% Fix anchor placement for figures with captions.
\usepackage{hypcap}% it must be loaded after hyperref.
% Set up styles of URL: it should be placed after hyperref.
\urlstyle{same}


\usepackage{sphinxmessages}




\title{Documentation CH552}
\date{Apr 26, 2024}
\release{0.0.1}
\author{Cesar Bautista}
\newcommand{\sphinxlogo}{\vbox{}}
\renewcommand{\releasename}{Release}
\makeindex
\begin{document}

\ifdefined\shorthandoff
  \ifnum\catcode`\=\string=\active\shorthandoff{=}\fi
  \ifnum\catcode`\"=\active\shorthandoff{"}\fi
\fi

\pagestyle{empty}
\sphinxmaketitle
\pagestyle{plain}
\sphinxtableofcontents
\pagestyle{normal}
\phantomsection\label{\detokenize{index::doc}}


\sphinxAtStartPar
\sphinxstylestrong{Cocket Nova}

\sphinxAtStartPar
The Coket Nova guide is an instructional manual for utilizing the compiler SDCC. It serves various purposes, allowing users to explore different features and obtain new projects and
configurations.
The Cocket Nova CH552 boards are incredibly easy to use. The projects focus on innovation and
present various alternatives for usage.

\noindent{\hspace*{\fill}\sphinxincludegraphics[width=0.800\linewidth]{{CH552_Sq}.png}\hspace*{\fill}}

\sphinxAtStartPar
Check out the  section for further information, including
installation instructions, and the \DUrole{xref,std,std-doc}{api} section for detailed

\begin{sphinxadmonition}{note}{Note:}
\sphinxAtStartPar
This project is under active development.
\end{sphinxadmonition}

\sphinxstepscope


\chapter{About}
\label{\detokenize{about:about}}\label{\detokenize{about::doc}}

\section{Describing}
\label{\detokenize{about:describing}}
\sphinxAtStartPar
This is an excellent guide for beginner programmers, focused on using the SDCC compiler in both Windows and Linux environments.
Here, you can find excellent references and examples along with comprehensive documentation focusing on developing technology for embedded systems.
This course covers everything from installation and setting up the compiler to managing project dependencies and developing code. It’s a valuable resource that will guide you through the development process using high\sphinxhyphen{}quality technology, ensuring long\sphinxhyphen{}lasting and robust projects.
\begin{quote}

\noindent{\hspace*{\fill}\sphinxincludegraphics[width=0.600\linewidth]{{CH552}.png}\hspace*{\fill}}
\end{quote}


\section{Requirements}
\label{\detokenize{about:requirements}}\begin{itemize}
\item {} 
\sphinxAtStartPar
Python (Package Installation and Environments)

\item {} 
\sphinxAtStartPar
Utilization of Operating System Controllers

\item {} 
\sphinxAtStartPar
Understanding of Basic Electronics

\end{itemize}

\begin{sphinxadmonition}{tip}{Tip:}
\sphinxAtStartPar
This  guide is tailored for individuals with a basic understanding of programming and electronics. It’s ideal for those interested in diving into embedded systems and programming microcontrollers.
\end{sphinxadmonition}


\section{PinOut}
\label{\detokenize{about:pinout}}
\sphinxAtStartPar
The CH552 microcontroller development board has a total of 16 pins, each with a specific function. The following is a list of the pins and their respective functions:

\noindent{\hspace*{\fill}\sphinxincludegraphics[width=0.800\linewidth]{{PinOut_CH552}.jpg}\hspace*{\fill}}

\sphinxstepscope


\chapter{CH55x with SDCC and Ubuntu 23.10}
\label{\detokenize{install_linux:ch55x-with-sdcc-and-ubuntu-23-10}}\label{\detokenize{install_linux::doc}}

\section{Ubuntu 23.10 Installer Environment}
\label{\detokenize{install_linux:ubuntu-23-10-installer-environment}}
\begin{sphinxadmonition}{note}{Note:}
\sphinxAtStartPar
This project is under active development. The information provided here is subject to change.
\end{sphinxadmonition}

\sphinxAtStartPar
Update the operating system:

\begin{sphinxVerbatim}[commandchars=\\\{\}]
\PYG{n}{sudo} \PYG{n}{apt} \PYG{n}{update}
\end{sphinxVerbatim}

\sphinxAtStartPar
Install \sphinxtitleref{make} and \sphinxtitleref{binutils}:

\begin{sphinxVerbatim}[commandchars=\\\{\}]
\PYG{n}{sudo} \PYG{n}{apt} \PYG{n}{install} \PYG{n}{make}
\PYG{n}{sudo} \PYG{n}{apt} \PYG{n}{install} \PYG{n}{binutils}
\end{sphinxVerbatim}

\sphinxAtStartPar
Clone the examples with the main code:

\begin{sphinxVerbatim}[commandchars=\\\{\}]
\PYG{n}{git} \PYG{n}{clone} \PYG{n}{https}\PYG{p}{:}\PYG{o}{/}\PYG{o}{/}\PYG{n}{github}\PYG{o}{.}\PYG{n}{com}\PYG{o}{/}\PYG{n}{Cesarbautista10}\PYG{o}{/}\PYG{n}{CH55x\PYGZus{}SDCC\PYGZus{}Examples}\PYG{o}{.}\PYG{n}{git}
\end{sphinxVerbatim}

\sphinxAtStartPar
Navigate to the path:

\begin{sphinxVerbatim}[commandchars=\\\{\}]
\PYG{n}{cd} \PYG{o}{\PYGZti{}}\PYG{o}{/}\PYG{n}{CH55x\PYGZus{}SDCC\PYGZus{}Examples}\PYG{o}{/}\PYG{n}{Software}\PYG{o}{/}\PYG{n}{examples}\PYG{o}{/}\PYG{l+m+mf}{0.}\PYGZbs{} \PYG{n}{Blink}\PYG{o}{/}
\end{sphinxVerbatim}

\sphinxAtStartPar
Execute the command:

\begin{sphinxVerbatim}[commandchars=\\\{\}]
\PYG{n}{make} \PYG{n}{help}
\end{sphinxVerbatim}

\sphinxAtStartPar
You will see:

\begin{sphinxVerbatim}[commandchars=\\\{\}]
\PYG{n}{Use} \PYG{n}{the} \PYG{n}{following} \PYG{n}{commands}\PYG{p}{:}
\PYG{n}{make} \PYG{n+nb}{all}     \PYG{n+nb}{compile}\PYG{p}{,} \PYG{n}{build}\PYG{p}{,} \PYG{o+ow}{and} \PYG{n}{keep} \PYG{n+nb}{all} \PYG{n}{files}
\PYG{n}{make} \PYG{n+nb}{hex}     \PYG{n+nb}{compile} \PYG{o+ow}{and} \PYG{n}{build} \PYG{n}{blink}\PYG{o}{.}\PYG{n}{hex}
\PYG{n}{make} \PYG{n+nb}{bin}     \PYG{n+nb}{compile} \PYG{o+ow}{and} \PYG{n}{build} \PYG{n}{blink}\PYG{o}{.}\PYG{n}{bin}
\PYG{n}{make} \PYG{n}{flash}   \PYG{n+nb}{compile}\PYG{p}{,} \PYG{n}{build}\PYG{p}{,} \PYG{o+ow}{and} \PYG{n}{upload} \PYG{n}{blink}\PYG{o}{.}\PYG{n}{bin} \PYG{n}{to} \PYG{n}{the} \PYG{n}{device}
\PYG{n}{make} \PYG{n}{clean}   \PYG{n}{remove} \PYG{n+nb}{all} \PYG{n}{build} \PYG{n}{files}
\end{sphinxVerbatim}

\sphinxAtStartPar
Connect a device with the BOOT button pressed:

\begin{sphinxVerbatim}[commandchars=\\\{\}]
\PYG{n}{lsusb}
\end{sphinxVerbatim}

\sphinxAtStartPar
The device will be shown with this description:

\begin{sphinxVerbatim}[commandchars=\\\{\}]
\PYG{n}{Descriptor}
\end{sphinxVerbatim}


\section{Install pyusb}
\label{\detokenize{install_linux:install-pyusb}}
\sphinxAtStartPar
Verify the installation with \sphinxtitleref{python \textendash{}version}. If not installed, run:

\begin{sphinxVerbatim}[commandchars=\\\{\}]
\PYG{n}{sudo} \PYG{n}{apt} \PYG{n}{install} \PYG{n}{python3}\PYG{o}{\PYGZhy{}}\PYG{n}{pip}
\end{sphinxVerbatim}

\sphinxAtStartPar
Then verify the installation:

\begin{sphinxVerbatim}[commandchars=\\\{\}]
\PYG{n}{python3} \PYG{o}{\PYGZhy{}}\PYG{n}{m} \PYG{n}{pip} \PYG{n}{show} \PYG{n}{pyusb}
\end{sphinxVerbatim}


\section{Error with pip}
\label{\detokenize{install_linux:error-with-pip}}
\sphinxAtStartPar
If you encounter this error, we recommend installing the Python environment:

\begin{sphinxVerbatim}[commandchars=\\\{\}]
\PYG{n}{sudo} \PYG{n}{apt} \PYG{n}{install} \PYG{n}{python3}\PYG{o}{\PYGZhy{}}\PYG{n}{venv}
\end{sphinxVerbatim}

\sphinxAtStartPar
Create an environment:

\begin{sphinxVerbatim}[commandchars=\\\{\}]
\PYG{n}{python3} \PYG{o}{\PYGZhy{}}\PYG{n}{m} \PYG{n}{venv} \PYG{o}{.}\PYG{n}{venv}
\end{sphinxVerbatim}

\sphinxAtStartPar
Activate the environment:

\begin{sphinxVerbatim}[commandchars=\\\{\}]
\PYG{n}{source} \PYG{o}{.}\PYG{n}{venv}\PYG{o}{/}\PYG{n+nb}{bin}\PYG{o}{/}\PYG{n}{activate}
\end{sphinxVerbatim}

\sphinxAtStartPar
And install \sphinxtitleref{pyusb}:

\begin{sphinxVerbatim}[commandchars=\\\{\}]
\PYG{n}{pip} \PYG{n}{install} \PYG{n}{pyusb}
\end{sphinxVerbatim}

\sphinxstepscope


\chapter{CH55x with SDCC and Windows}
\label{\detokenize{install_windows:ch55x-with-sdcc-and-windows}}\label{\detokenize{install_windows::doc}}

\section{Compiler Installation}
\label{\detokenize{install_windows:compiler-installation}}
\sphinxAtStartPar
Follow the steps below to install the necessary tools
\begin{itemize}
\item {} \begin{description}
\sphinxlineitem{Installing Git for Windows}
\sphinxAtStartPar
Download and install \sphinxhref{https://git-scm.com/downloads}{Git} for Windows from the official Git website.

\end{description}

\item {} \begin{description}
\sphinxlineitem{Installing SDCC}
\sphinxAtStartPar
Download and install the latest version of SDCC. You can find the latest version on the SDCC downloads page.
\sphinxhref{https://sourceforge.net/projects/sdcc/}{here a alternative}

\end{description}

\item {} \begin{description}
\sphinxlineitem{Installing MinGW}
\sphinxAtStartPar
Install MinGW, which is a set of tools for software development on Windows. You can download the installer from the official MinGW website.
\sphinxhref{https://sourceforge.net/projects/mingw/}{here a alternative}

\end{description}

\item {} \begin{description}
\sphinxlineitem{Installig Zadig}
\sphinxAtStartPar
Download the latest version of \sphinxhref{https://zadig.akeo.ie/}{Zadig}. You can download in a official website.

\end{description}

\end{itemize}


\section{Configuring MAKE}
\label{\detokenize{install_windows:configuring-make}}
\sphinxAtStartPar
Remember that for Windows operating systems, an extra step is necessary, which is to open the environment variable \sphinxhyphen{}\textgreater{} Edit environment variable:

\begin{sphinxVerbatim}[commandchars=\\\{\}]
\PYG{n}{C}\PYG{p}{:}\PYGZbs{}\PYG{n}{MinGW}\PYGZbs{}\PYG{n+nb}{bin}
\end{sphinxVerbatim}

\sphinxAtStartPar
Locate the file

\noindent{\hspace*{\fill}\sphinxincludegraphics[width=1.000\linewidth]{{make_file}.png}\hspace*{\fill}}

\sphinxAtStartPar
Rename it:

\noindent{\hspace*{\fill}\sphinxincludegraphics[width=1.000\linewidth]{{rename}.png}\hspace*{\fill}}

\sphinxAtStartPar
path:

\begin{sphinxVerbatim}[commandchars=\\\{\}]
\PYG{n}{C}\PYG{p}{:}\PYGZbs{}\PYG{n}{MinGW}\PYGZbs{}\PYG{n+nb}{bin}\PYGZbs{}\PYG{n}{make}
\end{sphinxVerbatim}

\noindent{\hspace*{\fill}\sphinxincludegraphics[width=1.000\linewidth]{{var_env}.png}\hspace*{\fill}}


\section{Install pyusb}
\label{\detokenize{install_windows:install-pyusb}}
\sphinxAtStartPar
Verify the installation with \sphinxtitleref{python \textendash{}version}.

\sphinxAtStartPar
If not installed, download and install:
\sphinxhyphen{} \sphinxhref{https://www.python.org/downloads/}{Python}

\sphinxAtStartPar
Use pip for install py usb:

\begin{sphinxVerbatim}[commandchars=\\\{\}]
\PYG{n}{pip} \PYG{n}{install} \PYG{n}{pyusb}
\end{sphinxVerbatim}


\section{Update driver}
\label{\detokenize{install_windows:update-driver}}
\sphinxAtStartPar
The current loading tool can utilize the default driver and coexist with the official WCHISPTool. In case the driver encounters issues, it is advisable to switch the driver version to libusb\sphinxhyphen{}win32 using Zadig.

\noindent{\hspace*{\fill}\sphinxincludegraphics[width=1.000\linewidth]{{driver}.png}\hspace*{\fill}}

\sphinxstepscope


\chapter{Compile and Flash CH55x with SDCC}
\label{\detokenize{compile:compile-and-flash-ch55x-with-sdcc}}\label{\detokenize{compile::doc}}

\section{Running a Program}
\label{\detokenize{compile:running-a-program}}
\sphinxAtStartPar
To run the program, use a bash terminal.

\sphinxAtStartPar
Clone the examples with the main code:

\begin{sphinxVerbatim}[commandchars=\\\{\}]
\PYG{n}{git} \PYG{n}{clone} \PYG{n}{https}\PYG{p}{:}\PYG{o}{/}\PYG{o}{/}\PYG{n}{github}\PYG{o}{.}\PYG{n}{com}\PYG{o}{/}\PYG{n}{Cesarbautista10}\PYG{o}{/}\PYG{n}{ESE\PYGZus{}CH552\PYGZus{}Examples\PYGZus{}C}\PYG{o}{.}\PYG{n}{git}
\end{sphinxVerbatim}

\sphinxAtStartPar
Navigate to the path:

\begin{sphinxVerbatim}[commandchars=\\\{\}]
\PYG{n}{cd} \PYG{o}{\PYGZti{}}\PYG{o}{/}\PYG{n}{CH55x\PYGZus{}SDCC\PYGZus{}Examples}\PYG{o}{/}\PYG{n}{Software}\PYG{o}{/}\PYG{n}{examples}\PYG{o}{/}\PYG{n}{Blink}\PYG{o}{/}
\end{sphinxVerbatim}

\sphinxAtStartPar
Connect a device with the BOOT button pressed.

\sphinxAtStartPar
Execute the command:

\begin{sphinxVerbatim}[commandchars=\\\{\}]
\PYG{n}{make} \PYG{n+nb}{all}
\end{sphinxVerbatim}

\sphinxAtStartPar
This will compile the project and generate files with the following extensions:

\noindent{\hspace*{\fill}\sphinxincludegraphics[width=0.800\linewidth]{{files}.png}\hspace*{\fill}}


\section{Flashing the Program}
\label{\detokenize{compile:flashing-the-program}}
\sphinxAtStartPar
Connect a device and press the BOOT button, then write the command:

\begin{sphinxVerbatim}[commandchars=\\\{\}]
\PYG{n}{make} \PYG{n}{flash}
\end{sphinxVerbatim}

\sphinxAtStartPar
If the project is successful, the code will generate a blinking effect as shown below:

\begin{figure}[htbp]
\centering

\noindent\sphinxincludegraphics[width=0.800\linewidth]{{led}.png}
\end{figure}

\sphinxstepscope


\chapter{General Board Control}
\label{\detokenize{generalboardcontrol:general-board-control}}\label{\detokenize{generalboardcontrol::doc}}
\sphinxAtStartPar
The CH552, characterized by its compact size, native USB connectivity, and 16 KB memory (with 14 KB usable), enables the creation of simple yet effective programs. This allows for greater control in implementing various applications. The choice of this microcontroller is based on its affordability, ease of connection, and compatibility with various operating systems.


\section{Recommended Operating Conditions}
\label{\detokenize{generalboardcontrol:recommended-operating-conditions}}

\begin{savenotes}\sphinxattablestart
\sphinxthistablewithglobalstyle
\centering
\sphinxcapstartof{table}
\sphinxthecaptionisattop
\sphinxcaption{Recommended Operating Conditions}\label{\detokenize{generalboardcontrol:id1}}
\sphinxaftertopcaption
\begin{tabular}[t]{\X{25}{100}\X{25}{100}\X{50}{100}}
\sphinxtoprule
\sphinxstyletheadfamily 
\sphinxAtStartPar
Symbol
&\sphinxstyletheadfamily 
\sphinxAtStartPar
Description
&\sphinxstyletheadfamily 
\sphinxAtStartPar
Range
\\
\sphinxmidrule
\sphinxtableatstartofbodyhook
\sphinxAtStartPar
VUSB
&
\sphinxAtStartPar
Voltage supply via USB
&
\sphinxAtStartPar
3.14 to 5.255 V
\\
\sphinxhline
\sphinxAtStartPar
VIn
&
\sphinxAtStartPar
Voltage supply from pins
&
\sphinxAtStartPar
2.7 to 5.5 V
\\
\sphinxhline
\sphinxAtStartPar
Top
&
\sphinxAtStartPar
Operating temperature
&
\sphinxAtStartPar
\sphinxhyphen{}40 to 85 °C
\\
\sphinxbottomrule
\end{tabular}
\sphinxtableafterendhook\par
\sphinxattableend\end{savenotes}


\section{Operating Mode}
\label{\detokenize{generalboardcontrol:operating-mode}}

\subsection{Voltage Selector}
\label{\detokenize{generalboardcontrol:voltage-selector}}
\sphinxAtStartPar
The development board utilizes a clever voltage selector system consisting of three pins and a jumper switch. The configuration of these pins determines the operating voltage of the board. By connecting the central pin to the +5V pin via the jumper, the board operates at 5V. On the other hand, by connecting the central pin to the +3.3V pin, the APK2112K regulator is activated, powering the board at 3.3V. It is crucial to ensure that the jumper switch is in the correct position according to the desired voltage to avoid possible damage to modules, components, and the board itself.

\noindent{\hspace*{\fill}\sphinxincludegraphics{{selector}.png}\hspace*{\fill}}


\subsection{JST Connectors}
\label{\detokenize{generalboardcontrol:jst-connectors}}
\sphinxAtStartPar
The board features two 1mm JST connectors, linked to different pins. The first connector directly connects to GPIO 3.0 and 3.1 of the microcontroller, while the second one is linked to pins 3.2 and 1.5. Both connectors operate in parallel to the selected power supply voltage via the jumper switch. These connectors are compatible with QWIIC, STEMMA QT, or similar pin distribution protocols. It is essential to verify that the selector voltage matches the system voltage to avoid circuit damage. Additionally, these connectors allow the board to be powered and offer functionalities such as PWM and serial communication.

\noindent{\hspace*{\fill}\sphinxincludegraphics{{jst}.png}\hspace*{\fill}}


\subsection{Built\sphinxhyphen{}In LEDs}
\label{\detokenize{generalboardcontrol:built-in-leds}}
\sphinxAtStartPar
The board features two LEDs directly linked to the microcontroller. The first one is connected to pin 3.4, while the second one is a Neopixel LED connected to pin 3.3. This Neopixel provides an output with two headers, one connected to the data output and the other to the board’s ground, allowing for the external connection of more LEDs. To use this output, simply connect the DOUT pin to the DIN pin of the next LED in the row. As for power supply, you can use the VCC pin, provided that the external LEDs can operate at this voltage. Otherwise, it will be necessary to power them using an external source.

\noindent{\hspace*{\fill}\sphinxincludegraphics{{neopixel}.png}\hspace*{\fill}}

\sphinxstepscope


\chapter{Analog to Digital Converter (ADC)}
\label{\detokenize{adc:analog-to-digital-converter-adc}}\label{\detokenize{adc::doc}}
\sphinxAtStartPar
The CH552 has four ADC channels, which can be used to read analog values from sensors. The ADC channels are multiplexed with the GPIO pins, so you can use any GPIO pin as an ADC input.
the distribution of the ADC channels is as follows:

\begin{sphinxVerbatim}[commandchars=\\\{\}]
\PYG{o}{\PYGZgt{}\PYGZgt{}} \PYG{n}{ADC0}\PYG{p}{:} \PYG{n}{P1}\PYG{l+m+mf}{.1}
\PYG{o}{\PYGZgt{}\PYGZgt{}} \PYG{n}{ADC1}\PYG{p}{:} \PYG{n}{P1}\PYG{l+m+mf}{.4}
\PYG{o}{\PYGZgt{}\PYGZgt{}} \PYG{n}{ADC2}\PYG{p}{:} \PYG{n}{P1}\PYG{l+m+mf}{.5}
\PYG{o}{\PYGZgt{}\PYGZgt{}} \PYG{n}{ADC3}\PYG{p}{:} \PYG{n}{P3}\PYG{l+m+mf}{.2}
\end{sphinxVerbatim}

\sphinxAtStartPar
The ADC has a resolution of 8 bits, which means it can read values from 0 to 255. The ADC can be configured to read values from 0 to 5V, or from 0 to 3.3V, depending on the VCC voltage:

\begin{sphinxVerbatim}[commandchars=\\\{\}]
\PYG{c+c1}{\PYGZsh{}include \PYGZdq{}src/config.h\PYGZdq{} // user configurations}
\PYG{c+c1}{\PYGZsh{}include \PYGZdq{}src/system.h\PYGZdq{} // system functions}
\PYG{c+c1}{\PYGZsh{}include \PYGZdq{}src/gpio.h\PYGZdq{}   // for GPIO}
\PYG{c+c1}{\PYGZsh{}include \PYGZdq{}src/delay.h\PYGZdq{}  // for delays}

\PYG{n}{void} \PYG{n}{main}\PYG{p}{(}\PYG{n}{void}\PYG{p}{)}
\PYG{p}{\PYGZob{}}
    \PYG{n}{CLK\PYGZus{}config}\PYG{p}{(}\PYG{p}{)}\PYG{p}{;}
    \PYG{n}{DLY\PYGZus{}ms}\PYG{p}{(}\PYG{l+m+mi}{5}\PYG{p}{)}\PYG{p}{;}

    \PYG{n}{ADC\PYGZus{}input}\PYG{p}{(}\PYG{n}{PIN\PYGZus{}ADC}\PYG{p}{)}\PYG{p}{;}

    \PYG{n}{ADC\PYGZus{}enable}\PYG{p}{(}\PYG{p}{)}\PYG{p}{;}
    \PYG{k}{while} \PYG{p}{(}\PYG{l+m+mi}{1}\PYG{p}{)}
    \PYG{p}{\PYGZob{}}
        \PYG{n+nb}{int} \PYG{n}{data} \PYG{o}{=} \PYG{n}{ADC\PYGZus{}read}\PYG{p}{(}\PYG{p}{)}\PYG{p}{;} \PYG{o}{/}\PYG{o}{/} \PYG{n}{Assuming} \PYG{n}{ADC\PYGZus{}read}\PYG{p}{(}\PYG{p}{)} \PYG{n}{returns} \PYG{n}{an} \PYG{n+nb}{int}
    \PYG{p}{\PYGZcb{}}
\PYG{p}{\PYGZcb{}}
\end{sphinxVerbatim}

\sphinxstepscope


\chapter{PWM (Pulse Width Modulation)}
\label{\detokenize{pwm:pwm-pulse-width-modulation}}\label{\detokenize{pwm::doc}}
\sphinxAtStartPar
The PWM module is used to generate a PWM signal on a pin. The PWM signal is generated by changing the duty cycle of the signal. The duty cycle is the ratio of the time the signal is high to the total time of the signal. The PWM module can be used to control the brightness of an LED, the speed of a motor, or the position of a servo motor.

\sphinxAtStartPar
The board contain two PWM pins, which are PIN\_PWM and PIN\_PWM2. The PWM module can be used to generate a PWM signal on these pins:

\begin{sphinxVerbatim}[commandchars=\\\{\}]
\PYG{n}{PWM} \PYG{l+m+mi}{1} \PYG{p}{:} \PYG{n}{P30}\PYG{o}{/}\PYG{n}{P15}
\PYG{n}{PWM} \PYG{l+m+mi}{2} \PYG{p}{:} \PYG{n}{P31}\PYG{o}{/}\PYG{n}{P34}
\end{sphinxVerbatim}

\sphinxAtStartPar
Some of the functions provided by the PWM module are:

\begin{sphinxVerbatim}[commandchars=\\\{\}]
\PYG{c+c1}{\PYGZsh{}define MIN\PYGZus{}COUNTER 10}
\PYG{c+c1}{\PYGZsh{}define MAX\PYGZus{}COUNTER 254}
\PYG{c+c1}{\PYGZsh{}define STEP\PYGZus{}SIZE   10}

\PYG{n}{void} \PYG{n}{change\PYGZus{}pwm}\PYG{p}{(}\PYG{n+nb}{int} \PYG{n}{hex\PYGZus{}value}\PYG{p}{)}
\PYG{p}{\PYGZob{}}
    \PYG{n}{PWM\PYGZus{}write}\PYG{p}{(}\PYG{n}{PIN\PYGZus{}PWM}\PYG{p}{,} \PYG{n}{hex\PYGZus{}value}\PYG{p}{)}\PYG{p}{;}
\PYG{p}{\PYGZcb{}}
\PYG{n}{void} \PYG{n}{main}\PYG{p}{(}\PYG{n}{void}\PYG{p}{)}
\PYG{p}{\PYGZob{}}
    \PYG{n}{CLK\PYGZus{}config}\PYG{p}{(}\PYG{p}{)}\PYG{p}{;}
    \PYG{n}{DLY\PYGZus{}ms}\PYG{p}{(}\PYG{l+m+mi}{5}\PYG{p}{)}\PYG{p}{;}
    \PYG{n}{PWM\PYGZus{}set\PYGZus{}freq}\PYG{p}{(}\PYG{l+m+mi}{1}\PYG{p}{)}\PYG{p}{;}
    \PYG{n}{PIN\PYGZus{}output}\PYG{p}{(}\PYG{n}{PIN\PYGZus{}PWM}\PYG{p}{)}\PYG{p}{;}
    \PYG{n}{PWM\PYGZus{}start}\PYG{p}{(}\PYG{n}{PIN\PYGZus{}PWM}\PYG{p}{)}\PYG{p}{;}
    \PYG{n}{PWM\PYGZus{}write}\PYG{p}{(}\PYG{n}{PIN\PYGZus{}PWM}\PYG{p}{,} \PYG{l+m+mi}{0}\PYG{p}{)}\PYG{p}{;}
\PYG{k}{while} \PYG{p}{(}\PYG{l+m+mi}{1}\PYG{p}{)}
\PYG{p}{\PYGZob{}}
    \PYG{k}{for} \PYG{p}{(}\PYG{n+nb}{int} \PYG{n}{i} \PYG{o}{=} \PYG{n}{MIN\PYGZus{}COUNTER}\PYG{p}{;} \PYG{n}{i} \PYG{o}{\PYGZlt{}} \PYG{n}{MAX\PYGZus{}COUNTER}\PYG{p}{;} \PYG{n}{i}\PYG{o}{+}\PYG{o}{=}\PYG{n}{STEP\PYGZus{}SIZE}\PYG{p}{)}
    \PYG{p}{\PYGZob{}}
        \PYG{n}{change\PYGZus{}pwm}\PYG{p}{(}\PYG{n}{i}\PYG{p}{)}\PYG{p}{;}
        \PYG{n}{DLY\PYGZus{}ms}\PYG{p}{(}\PYG{l+m+mi}{20}\PYG{p}{)}\PYG{p}{;}
    \PYG{p}{\PYGZcb{}}
    \PYG{k}{for} \PYG{p}{(}\PYG{n+nb}{int} \PYG{n}{i} \PYG{o}{=} \PYG{n}{MAX\PYGZus{}COUNTER}\PYG{p}{;} \PYG{n}{i} \PYG{o}{\PYGZgt{}} \PYG{n}{MIN\PYGZus{}COUNTER}\PYG{p}{;} \PYG{n}{i}\PYG{o}{\PYGZhy{}}\PYG{o}{=}\PYG{n}{STEP\PYGZus{}SIZE}\PYG{p}{)}
    \PYG{p}{\PYGZob{}}
        \PYG{n}{change\PYGZus{}pwm}\PYG{p}{(}\PYG{n}{i}\PYG{p}{)}\PYG{p}{;}
        \PYG{n}{DLY\PYGZus{}ms}\PYG{p}{(}\PYG{l+m+mi}{20}\PYG{p}{)}\PYG{p}{;}
    \PYG{p}{\PYGZcb{}}
\PYG{p}{\PYGZcb{}}
\PYG{p}{\PYGZcb{}}
\end{sphinxVerbatim}

\sphinxstepscope


\chapter{I2C (Inter\sphinxhyphen{}Integrated Circuit)}
\label{\detokenize{i2c:i2c-inter-integrated-circuit}}\label{\detokenize{i2c::doc}}
\sphinxAtStartPar
I2C is a serial communication protocol, so data is transferred bit by bit along a single wire (the SDA line). The SCL line is used to synchronize the data transfer. The I2C protocol is a master\sphinxhyphen{}slave protocol, which means that the communication is always initiated by the master device. A master device can communicate with one or multiple slave devices. The master device generates the clock signal and initiates the data transfer. The slave devices are addressed by the master device, and they respond to the master device based on the address they are assigned. The I2C protocol supports multiple devices on the same bus, and each device has a unique address. This allows multiple devices to communicate with each other using the same bus.

\begin{sphinxadmonition}{note}{Note:}
\sphinxAtStartPar
The microcontroller use bit banging to communicate with the I2C devices.
\end{sphinxadmonition}

\sphinxAtStartPar
All pin configurations are defined in the config.h file. The I2C pins are defined as follows:

\begin{sphinxVerbatim}[commandchars=\\\{\}]
\PYG{c+c1}{\PYGZsh{}define I2C\PYGZus{}SCL\PYGZus{}PIN  P32}
\PYG{c+c1}{\PYGZsh{}define I2C\PYGZus{}SDA\PYGZus{}PIN  P30}
\end{sphinxVerbatim}


\section{OLED Display SD1306}
\label{\detokenize{i2c:oled-display-sd1306}}
\sphinxAtStartPar
The OLED display is a 128x64 pixel monochrome display that uses the I2C protocol for communication. The display has a built\sphinxhyphen{}in controller that handles the display refresh and data transfer. The display is controlled by the microcontroller using the I2C protocol. The display has a resolution of 128x64 pixels, which allows it to display text, graphics, and images. The display is monochrome, which means that it can only display one color (white) on a black background. The display has a built\sphinxhyphen{}in controller that handles the display refresh and data transfer. The display is controlled by the microcontroller using the I2C protocol. The display has a resolution of 128x64 pixels, which allows it to display text, graphics, and images. The display is monochrome, which means that it can only display one color (white) on a black background:

\begin{sphinxVerbatim}[commandchars=\\\{\}]
\PYG{n}{void} \PYG{n}{beep}\PYG{p}{(}\PYG{n}{void}\PYG{p}{)} \PYG{p}{\PYGZob{}}
\PYG{n}{uint8\PYGZus{}t} \PYG{n}{i}\PYG{p}{;}
\PYG{k}{for}\PYG{p}{(}\PYG{n}{i}\PYG{o}{=}\PYG{l+m+mi}{255}\PYG{p}{;} \PYG{n}{i}\PYG{p}{;} \PYG{n}{i}\PYG{o}{\PYGZhy{}}\PYG{o}{\PYGZhy{}}\PYG{p}{)} \PYG{p}{\PYGZob{}}
    \PYG{n}{PIN\PYGZus{}low}\PYG{p}{(}\PYG{n}{PIN\PYGZus{}BUZZER}\PYG{p}{)}\PYG{p}{;}
    \PYG{n}{DLY\PYGZus{}us}\PYG{p}{(}\PYG{l+m+mi}{125}\PYG{p}{)}\PYG{p}{;}
    \PYG{n}{PIN\PYGZus{}high}\PYG{p}{(}\PYG{n}{PIN\PYGZus{}BUZZER}\PYG{p}{)}\PYG{p}{;}
    \PYG{n}{DLY\PYGZus{}us}\PYG{p}{(}\PYG{l+m+mi}{125}\PYG{p}{)}\PYG{p}{;}
\PYG{p}{\PYGZcb{}}
\PYG{p}{\PYGZcb{}}

\PYG{n}{void} \PYG{n}{main}\PYG{p}{(}\PYG{n}{void}\PYG{p}{)} \PYG{p}{\PYGZob{}}
\PYG{o}{/}\PYG{o}{/} \PYG{n}{Setup}
\PYG{n}{CLK\PYGZus{}config}\PYG{p}{(}\PYG{p}{)}\PYG{p}{;}                           \PYG{o}{/}\PYG{o}{/} \PYG{n}{configure} \PYG{n}{system} \PYG{n}{clock}
\PYG{n}{DLY\PYGZus{}ms}\PYG{p}{(}\PYG{l+m+mi}{5}\PYG{p}{)}\PYG{p}{;}                              \PYG{o}{/}\PYG{o}{/} \PYG{n}{wait} \PYG{k}{for} \PYG{n}{clock} \PYG{n}{to} \PYG{n}{stabilize}

\PYG{n}{OLED\PYGZus{}init}\PYG{p}{(}\PYG{p}{)}\PYG{p}{;}                            \PYG{o}{/}\PYG{o}{/} \PYG{n}{init} \PYG{n}{OLED}

\PYG{n}{OLED\PYGZus{}print}\PYG{p}{(}\PYG{l+s+s2}{\PYGZdq{}}\PYG{l+s+s2}{*  UNITelectronics  *}\PYG{l+s+s2}{\PYGZdq{}}\PYG{p}{)}\PYG{p}{;}
\PYG{n}{OLED\PYGZus{}print}\PYG{p}{(}\PYG{l+s+s2}{\PYGZdq{}}\PYG{l+s+s2}{\PYGZhy{}\PYGZhy{}\PYGZhy{}\PYGZhy{}\PYGZhy{}\PYGZhy{}\PYGZhy{}\PYGZhy{}\PYGZhy{}\PYGZhy{}\PYGZhy{}\PYGZhy{}\PYGZhy{}\PYGZhy{}\PYGZhy{}\PYGZhy{}\PYGZhy{}\PYGZhy{}\PYGZhy{}\PYGZhy{}\PYGZhy{}}\PYG{l+s+se}{\PYGZbs{}n}\PYG{l+s+s2}{\PYGZdq{}}\PYG{p}{)}\PYG{p}{;}
\PYG{n}{OLED\PYGZus{}print}\PYG{p}{(}\PYG{l+s+s2}{\PYGZdq{}}\PYG{l+s+s2}{Ready}\PYG{l+s+se}{\PYGZbs{}n}\PYG{l+s+s2}{\PYGZdq{}}\PYG{p}{)}\PYG{p}{;}
\PYG{n}{beep}\PYG{p}{(}\PYG{p}{)}\PYG{p}{;}
\PYG{k}{while}\PYG{p}{(}\PYG{l+m+mi}{1}\PYG{p}{)} \PYG{p}{\PYGZob{}}

\PYG{p}{\PYGZcb{}}
\PYG{p}{\PYGZcb{}}
\end{sphinxVerbatim}



\renewcommand{\indexname}{Index}
\printindex
\end{document}